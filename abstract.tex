\begin{abstract}

\hyphenation{α-ντι-κει-με-νο-στρε-φών εν-το-πι-σμό}

Η ανίχνευση και διόρθωση λαθών σε προγράμματα είναι μια διαδικασία που
καταναλώνει σημαντικό μέρος του χρόνου κάθε προγραμματιστή. Εργαλεία που
διευκολύνουν τον εντοπισμό λαθών είναι χρήσιμα τόσο στον περιορισμό των τελικών
λαθών, όσο και στην διαδικασία εντοπισμού και διόρθωσης κατά την ανάπτυξη του
προγράμματος. Στην παρουσα εργασία παρουσιάζεται η επέκταση των δυνατότητων του
\dr, ενός προγράμματος που χρησιμοποιεί στατική ανάλυση για την ανίχνευση λαθών 
σε προγράμματα στη γλώσσα \er, με την εισαγωγή \emph{τύπων τομής} που βελτιώνουν
σημαντικά την ακρίβεια του εργαλείου και την προσθήκη επιπλέον δυνατοτήτων σχετικών
με τον εντοπισμό λαθών στην χρήση των \emph{behaviours}, που αντιστοιχούν στις
\emph{αφηρημένες κλάσεις} των αντικειμενοστρεφών γλωσσών προγραμματισμού. Οι 
επεκτάσεις αυτές οδήγησαν στον εντοπισμό σημαντικών λαθών σε ήδη υπάρχοντα και
ενδελεχώς ελεγμένο κώδικα.

\begin{keywords}
Στατική ανάλυση, Συμπερασμός τύπων, Τύποι τομής, Αφηρημένες κλάσεις, \er, \dr
\end{keywords}
\end{abstract}

\begin{abstracteng}

Detection and correction of bugs consumes a significant amount of every
developer's time. Any tool designed with the intention to make this task easier
is useful to both minimize the final bugs present in the code and to detect and
correct them before release. This thesis describes the extension of \dr, a static 
analysis tool designed to find discrepancies in \er\ programs, with the
introduction of \emph{intersection types} that lead to impovements in its accuracy 
and the addition of a new module capable of finding errors in the use of
\emph{behaviours}, which correspond to \emph{abstract classes} as they appear in
object-oriented languages. These extensions lead to the detection of important
errors in already thoroughly tested code.

\begin{keywordseng}
Static analysis, Type inference, Intersection types, Behaviours, Abstract classes, \er, \dr
\end{keywordseng}
\end{abstracteng}
