\chapter{Related and Further work}

\section{Related work}
\label{sct:related_work}

There is almost no other known cases where tools need to make checks
like the ones implemented for \er's \emph{behaviours}, either in their
generic usage or in the detection of races through
them. Object-oriented languages that use the equivalent \emph{abstract
  classes}, \emph{virtual methods} and \emph{interfaces} have static
typing (C++, Java, OCaml, ...) which ensures the fitting of
implementations in every case.

On the subject of intersection types in a dynamically typed
programming language, only DRuby is a known analog and is presented
below.

The changes in the refinement procedure presented in Section
\ref{sct:intersect_refinement} are related with many formal approaches
to control flow analysis

\subsection{Diamondback Ruby (DRuby)}

Diamondback Ruby (DRuby)\cite{druby} is a recent tool that blends
Ruby's dynamic type system with a static typing discipline. It uses a
similar approach as \dr\ generating constraints for the variables then
applying a set of rewrite rules exhaustively. Intersection types are
included from the beginning in it's type system, but it cannot infer
them automatically. As a result they need to be annotated by the
developer placing them on the same level as \dr's contracts
(\emph{specs}).

\section{Further work}
\label{sct:further_work}

\subsection{Behaviours}
\subsubsection{Automatic bypass of API for race detection}

The ``bypass'' mechanism was designed to be extensible, allowing other
behaviour API's to be connected with the respective callbacks. Anyone
with a better understanding of the other behaviours may document the
rest OTP's behaviours easily by adding code in
\emph{dialyzer\_behaviours} module. It would be even better if this
connection was tranferred in each behaviour's file, as a new attribute
or as an extension on the \emph{callback} attribute that was
introduced in this thesis.

\subsection{Intersections}

\subsubsection{Negative Types}
\label{sct:negative_types}

The next logical extension to the type system would be negative
types. Examples would be \emph{``any term except the integer 42''} or
\emph{``any atom except a''}. With this infrastructure, when
\dr\ generates disjunctive lists it will be able to eliminate the
types already covered in previous clauses in the following ones. Thus
a ``catch-all'' will not have type \any\ but \emph{``anything but X, Y
  and Z''} where \emph{X, Y} and \emph{Z} will be the types already
covered by previous clauses.

\subsubsection{Tighter coupling between type and code}

Using normal form and sorting on the function's type has an impact on
the relation between the type of the function and the actual code that
generated it. Even though it was easy to find the cause of all the
warnings emitted when intersection types were used, it might be better
to narrow down a warning to a particular clause instead of the generic
pointer to the first line of the function.
