\chapter{Introduction}

\dr\ is one of the most widely-used tools in the development of
\er\ programs. Its name stands for DIscrepancy anALYZer for ERlang and
it does exactly that: in an inherently dynamically typed language such
as \er, \dr\ is able to detect many type related discrepancies using
static analysis. The initial version reported type errors using
success typings but subsequent extensions allowed for verification of
user contracts, detection of violations of the opaqueness of certain
abstract data types and recently even warnings about race
conditions~\cite{Races@PADL-10,
  springerlink:10.1007/978-3-540-30477-7_7, SuccessTypings@PPDP-06,
  opaques}.

Altough taking part in the development of such a tool is a pleasure in
itself, this thesis begun with a more concrete motivation: the
extension of \dr\ to detect discrepancies in the use of \er's
\emph{behaviours} (more on these in
Section~\ref{sct:behaviour_preliminaries}), including simple
type-related checks and enabling the new race detection analysis to
pass through them undistracted. This goal was accomplished easily, as
described in Chapter~\ref{chp:behaviours}, but the cost was that
\dr\ produced some false warnings that couldn't be avoided using the
existing type inference algorithm (see
Section~\ref{sct:behaviour_discrepancies}).

As this couldn't be tolerated (one of \dr's cornerstones is that it is
never wrong about a warning it emits) the second part of this thesis
came into focus. When one designs a tool that promises to detect
discrepancies in code soundly the main question that needs to be
addressed is not what to include but what to leave outside. One of the
greatest compromises made in \dr's initial design was that it would
work with a type system \textbf{without} \emph{intersection types} for
functions (see Section~\ref{sct:dialyzer_preliminaries}). This lead to
its inability to report glaring errors as the one presented in
Section~\ref{sct:intersection_preliminaries} and was the reason behind
the violation of soundness in behaviour analysis as well.

As in every happy story, this is no longer the case!
Chapters~\ref{chp:intersection_generate}
and~\ref{chp:intersection_usage} show how intersection types can be
generated and used. \dr\ emerges stronger than before, able to catch
both abuses of behaviours and a whole new range of actual errors in
code. What is more, further improvements are now easily attainable and
are presented in Section~\ref{sct:further_work}.
