\chapter{Preliminaries}
\label{chp:preliminaries}

\section{\er\ and OTP}
\label{sct:erlang_preliminaries}

\er\ is a strict, dynamically typed functional programming language
with support for concurrency, communication, distribution,
fault-tolerance, on-the-fly code reloading, automatic memory
management and support for multiple platforms~\cite{joe_book1}. The
number of areas where \er\ is actively used is increasing. However,
its primary application area is still in large-scale embedded control
systems developed by the telecom industry.

The main implementation of the language, the \er/OTP (Open Telecom
Platform) system from Ericsson, has been open source since 1998 and
has been used quite successfully both by Ericsson and by other
companies around the world to develop software for large commercial
applications.

Nowadays, applications written in the language are significant both in
number and in code size making \er\ one of the most industrially
relevant declarative languages.

\section{Behaviours}
\label{sct:behaviour_preliminaries}

As expected from an industrially used framework, OTP provides
components that make the use of the language easier. Using these
components developers are able to take advantage of the aforementioned
language features. An excellent guide on how to develop a
fault-tolerant, distributed and concurrent application is available to
help new developers learn the language quickly~\cite{des_princ}.

One of the key elements of the framework are the
\emph{behaviours}. These correspond to the \emph{abstract classes} or
\emph{interfaces} found in object-oriented languages, like Java, as
they divide the functionality of a component into a generic part (the
\emph{behaviour} module) and a specific part (the \emph{callback}
module).

OTP provides many behaviour modules. To implement a process such as a
server for example, the user only has to implement the callback module
which should export a pre-defined set of functions, the callback
functions. It is also possible for a developer to design his own
behaviour, either by splitting the functionality of the program or by
following a known design pattern~\cite{behaviours}.

\section{\dr}
\label{sct:dialyzer_preliminaries}

\dr\ is a static analysis tool included in the OTP since 2007. It can
detect a wide variety of discrepancies (i.e., type errors, software
defects such as exception-raising code, hidden failures, unsatisfiable
conditions, redundancies such as unreachable code, race conditions,
etc.) in single modules or entire applications. Dialyzer is totally
automatic, extremely easy to use and particularly successful in
identifying software defects which may be hidden in Erlang code,
especially in program paths which are not exercised by testing.

In the heart of \dr\ lies a soft typing system. Its purpose is
essentially to capture the biggest set of terms for which it can be
proven that type clashes will occur. The type signatures that
\dr\ infers, called \emph{success typings}, are the complement of that
set of terms. Success typings are an over-approximation to the set of
terms for which a function can evaluate: the domain of the signature
includes all possible values that the function could accept as
parameters, and its range includes all possible return values for this
domain. Success typings are guaranteed to capture all intended uses of
a function, along, perhaps, with some erroneous ones. Thus, any use of
a function that is incompatible with its success typing will
definitely fail. In effect, success typings approach the type
inference problem from a direction opposite to that of type systems
for statically typed languages.

For the actual workings of \dr\ more details can be found in the
relevant bibliography~\cite{Elli, type_system,
  springerlink:10.1007/978-3-540-30477-7_7, SuccessTypings@PPDP-06,
  GradualTyping@Erlang-08}. Only some key features will be presented
concisely as they are relevant with the modifications this thesis
describes.

\subsection{Analysis phases}

\dr\ operation can be split in two phases:

\begin{enumerate}
\item \textbf{Find success typings:} In this phase \dr\ traverses the
  code of every function included in the analysis and finds the
  success typing of each function. This requires several iterations of
  simple constraint solving and dataflow analysis. In the end, every
  function is assigned a final success typing.
\item \textbf{Emit warnings:} After the success typings have been
  fixed the code is traversed one more time. In this run a warning is
  emitted whenever a discrepancy is detected.
\end{enumerate}

\subsection{Refinement of success typings}

In the calculation of success typings, functions that are not
exported undergo a further refinement. The actual calls to these
functions are used to calculate the effective domain. This is
possible, as all the calls are located in the module under
analysis. Restricting the domain may tighten the final success typing
and possibly render some clauses unneeded. In Listing~\ref{lst:refine}
an example is provided. Even though \texttt{foo}'s initial type will be
calculated as $number \rightarrow number$ the final success typing
will be $42 \rightarrow 43$ allowing a stricter return type for $test$
as well (initially $number$, finally $43$).

\begin{console}{lst:refine}{Refining a local function}
-module(refine).
-export([test/0]).

test() ->
  foo(42).

foo(X) -> X + 1;
\end{console}

\subsection{Contracts}

\dr\ can take into account annotations placed by developers to further
restrict the success typings. These are called
$specs$~\cite{type_system} and are used for both type checking (by
\dr) and documentation (by \mbox{\textsc{Edoc}}). The contracts may
have more than one clauses, as it can be seen in the examples in
Listing~\ref{lst:specs}. These clauses though should not overlap (a
certain call must belong in exactly one of them).

\begin{console}{lst:specs}{Permitted and non-permitted specs}
%% Permitted spec with one clause
-spec foo(number()) -> number().

%% Permitted spec with two clauses
-spec foo(integer()) -> integer();
         (  float()) -> float().

%% Non-permitted spec. 'bar' is an atom and overlaps with the 2nd clause
-spec foo( 'bar') -> 'ok';
         (atom()) -> 'error'.
\end{console}

\section{Intersection types}
\label{sct:intersection_preliminaries}

Intersection types are types describing values that belong to both of
two other given types. For example, in most implementations of C the
signed char has range -128 to 127 and the unsigned char has range 0 to
255, so the intersection type of these two types would have range 0 to
127. Such an intersection type could be safely passed into functions
expecting either signed or unsigned chars, because it is compatible
with both types.

Most modern statically typed programming languages support overloaded
functions. Such functions execute different code depending on the type
of the arguments they receive. Intersection types are useful for
describing the type of such functions: an example is a function with
type $ Int \rightarrow Int | Float \rightarrow Float $. This function
will return $Int$ if called with an $Int$ argument and a $Float$ if
called with $Float$. Type checking is an essential part of the
semantic analysis of a statically typed language's compiler.

However, \er\ is a dynamically typed language, so type checking is not
part of the compilation. What is more, the relevant tool, \dr, was
designed without intersection types. In
Listing~\ref{lst:trivial_intersection} we see a simple overloaded
function, \texttt{foo}. \dr\ using \emph{success typings} will find
that the function succeeds for arguments $a$ or $b$ and the return may
be $1$ or $2$. Lacking intersection types, this will be expressed as
$a | b \rightarrow 1 | 2 $. This is an overapproximation because
\texttt{foo} will never return 2 if the argument is $a$ and
vice-versa. This will cause the error presented in the combination of
\texttt{foo} with the other functions in the example to pass
undetected \footnote{In fact this will happen only if \texttt{buz} is
  exported. Otherwise the refinement described in
  Section~\ref{sct:dialyzer_preliminaries} will find that buz will be
  called with $b$ only and will refine \texttt{foo}'s type with this
  info, catching the error.}.

\begin{console}{lst:trivial_intersection}{Trivial error that can be detected with the use of intersection types}
  foo(a) -> 1;
  foo(b) -> 2.

  bar(1) -> ok.

  buz(X) ->
    bar(foo(X)).

  test() ->
    buz(b).
\end{console}

The goal of the second part of this thesis is the inference of
intersection types and is presented in
Chapters~\ref{chp:intersection_generate}
and~\ref{chp:intersection_usage}.
