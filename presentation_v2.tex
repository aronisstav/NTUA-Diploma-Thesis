\documentclass{beamer}

\usepackage[cm-default]{fontspec}
\usepackage{xunicode}
\usepackage{xltxtra}
\usepackage{multirow}
\usepackage{graphicx}

\setmainfont[Mapping=tex-text]{DejaVu Sans}

\usetheme{Pittsburgh}
\useinnertheme{rounded}
\usefonttheme{serif}
\usecolortheme{beaver}

\setbeamertemplate{navigation symbols}{}
\setbeamertemplate{footline}{
  \leavevmode
  \hbox{
    \begin{beamercolorbox}[wd=.5\paperwidth,ht=2.5ex,dp=1.125ex,right]
      {author in head/foot}
      \usebeamerfont{title in head/foot}
      \insertshortauthor
      \hspace{.3cm}
    \end{beamercolorbox}%
    
    \begin{beamercolorbox}[wd=.40\paperwidth,ht=2.5ex,dp=1.125ex,left]
      {title in head/foot}
      \usebeamerfont{author in head/foot}\
      \hspace{.3cm}
      \insertshorttitle
    \end{beamercolorbox}
    
    \begin{beamercolorbox}[wd=.10\paperwidth,ht=2.5ex,dp=1.125ex,center]
      {title in head/foot}
      \usebeamerfont{author in head/foot}
      \insertframenumber/\inserttotalframenumber
    \end{beamercolorbox}
  }
  \vskip0pt
}

\usepackage{listings}
\lstdefinelanguage{erlang}{
    morekeywords={
	% Reserved words
	after,and,andalso,band,begin,bnot,bor,bsl,bsr,bxor,case,catch,cond,div,
	end,fun,if,let,not,of,or,orelse,query,receive,rem,try,when,xor,
	% Common attributes
	behavior,behaviour,callback,compile,export,export_type,import,include,
	include_lib,module,opaque,record,spec,type,
	% Macro-related attributes
	define,ifdef,ifndef,else,endif,undef
    },
    morekeywords=[2]{
	% Process-related BIFs
	apply,exit,throw,get,put,erase,%error,
	% Useful general purpose BIFs
	%abs,min,max,trunc,round,size,bit_size,byte_size,tuple_size,element,
	%setelement,length,hd,tl,
	% Type test BIFs
	is_atom,is_binary,is_bitstring,is_boolean,is_float,is_function,
	is_integer,is_list,is_number,is_pid,is_port,is_record,is_reference,
	is_tuple,
	% Type conversion BIFs
	atom_to_binary,atom_to_list,binary_to_atom,binary_to_list,
	binary_to_term,bitstring_to_list,integer_to_list,list_to_atom,
	list_to_binary,list_to_bitstring,list_to_float,list_to_integer,
	list_to_pid,list_to_tuple,term_to_binary,tuple_to_list,float_to_list,
	pid_to_list
    },
    morekeywords=[3]{
	% Type names
	integer,non_neg_integer,pos_integer,neg_integer,float,atom,binary,
	bitstring,pid,port,reference,function,tuple,list,any,none,
	% Common atoms
	true,false,ok,error,undefined
    },
    otherkeywords={->,!},
    morecomment=[l]\%,
    morestring=[b]",
    morestring=[b]'
}[keywords,comments,strings]
\lstset{
    numbers=left,
    numberstyle=\tiny,
    basicstyle=\ttfamily\footnotesize,
    basewidth=0.59em,
    keywordstyle=[3]{},
    commentstyle=\itshape\footnotesize,
    tabsize=8,
    frame=single,
    frameround=tttt,
    showstringspaces=false,
    breaklines=false,
    captionpos=b,
    aboveskip=\bigskipamount,
    belowskip=\bigskipamount,
    escapechar=^
}
% Style options:
% numberstyle,basicstyle,identifierstyle,commentstyle,stringstyle
% keywordstyle=[1]{},keywordstyle=[2]{},directivestyle
% \small\tiny\footnotesize\itshape\ttfamily\bf
\lstnewenvironment{code}[2]{
    \nopagebreak
    \lstset{language=erlang,float=htbp,label={#1},caption={#2}}
}{}
\lstnewenvironment{console}[2]{
    \nopagebreak
    \lstset{float=htbp,label={#1},caption={#2},numbers=none}
}{}


\lstnewenvironment{code}[1]{
    \nopagebreak
    \lstset{title={#1},numbers=left}
}{}

\newenvironment{fulltable}[3]{
  \def\tempcaption{#2}
  \begin{table}[hbtp]
    \tiny
    \begin{center}
      \begin{tabular}[c]{#1}
}{
      \end{tabular}
    \end{center}
    \caption{\tempcaption}
  \end{table}
}

\newcommand{\er}{Erlang}
\newcommand{\st}{\emph{success typing}}
\newcommand{\sts}{\emph{success typings}}
\newcommand{\Sts}{\emph{Success typings}}

\title[Intersection type inference] {Intersection type inference for
  more accurate detection of defects in Erlang programs}
\author{Stavros Aronis}
\institute{Department of Electrical and Computer Engineering, National
  Technical University of Athens}

\AtBeginSection[]
{
  \begin{frame}<beamer>
    \frametitle{Overview}
    \tableofcontents[currentsection, hideothersubsections]
  \end{frame}
}

\begin{document}

\begin{frame}
  \titlepage
  \begin{center}
    \includegraphics[width=2cm]{pyrforos}
  \end{center}
\end{frame}

\section*{Prelude}

\begin{frame}[fragile]
  \frametitle{An elementary question...}
  What is the type of the following function?
\begin{code}{List length}
length     [] = 0
length (_:xs) = 1 + length xs
\end{code}
\pause
\begin{code}{Haskell}
[a] -> Int
\end{code}
\begin{code}{SML}
'a list -> int
\end{code}
\pause
\begin{code}{Erlang's Dialyzer (\st)}
[_] -> non_negative_integer()
\end{code}
\end{frame}

\begin{frame}[fragile]
  \frametitle{Does it really matter?}
\begin{code}{Example}
fun foo x = if (length x) < 0 then 0 else 1
\end{code}
  \pause
  \begin{itemize}
  \item Haskell and ML won't complain \pause
  \item Dialyzer will!
\begin{code}{Same example in Erlang}
foo(X) -> Length = length(X),
          if Length < 0 -> 0;
             true       -> 1
          end.

%% WARNING! %%
Guard test Length::non_neg_integer() < 0 
can never succeed
\end{code}
  \end{itemize}
\end{frame}

\begin{frame}[fragile]
  \frametitle{Setting up a crime scene...}
\begin{code}{Empty list check}
empty([]   ) -> true;
empty([_|_]) -> false.
\end{code}
\pause
\begin{code}{Dialyzer's \st}
[_] -> boolean()
\end{code}
\pause
\begin{code}{Sneakier example}
foo(X) -> case empty(X) of
            false -> ...;
            true  -> Length = length(X),
                     if Length =:= 0 -> ...;
                        true         -> ...
                     end
          end.
\end{code}
\end{frame}

\begin{frame}[fragile]
  \frametitle{It escaped!}
  \begin{itemize}
    \item Dialyzer missed that one... \pause
    \item ... but the information was there
\begin{code}{List length}
length     [] = 0
length (_:xs) = 1 + length xs
\end{code} \pause
    \item Meybe we can find an even better type? \pause
\begin{code}{Like this?}
[]      -> 0
[_,...] -> positive_integer()
\end{code}
  \end{itemize}
\end{frame}

\begin{frame}
  \frametitle{Overview}
  \tableofcontents[hidesubsections]
\end{frame}

\section{Success typings paradigm}

\subsection{Terminology}

\begin{frame}
  \frametitle{Terminology}
  \begin{itemize}
  \item \textbf{\Sts}\cite{success_typings} provide a
    simple, bottom-up approach to the problem of finding types for
    programs in a language that is dynamically typed \pause
  \item A \textbf{\st} for a function describes:
    \begin{itemize}
    \item all possible values that the function could accept as
      arguments (domain)
    \item all possible return values for this domain
    \end{itemize} \pause
    \begin{definition}
      A \st\ of a function $f$ is a type signature,
      $(\bar{\alpha}) \rightarrow \beta$ such that whenever an
      application $f(\bar{p})$ reduces to a value $v$, then $v \in
      \beta$ and $\bar{p} \in \bar{\alpha}$.
    \end{definition}
  \end{itemize}
\end{frame}

\begin{frame}
  \frametitle{Terminology}
  \Sts\ may therefore include:
  \begin{itemize}
  \item success arguments that will cause a call to fail
  \item return values that may never be returned
  \item overapproximations
  \end{itemize} \pause
  It is sure though that a call that does not fall into the domain
  will certainly fail and a value not predicted will never be
  returned!
\end{frame}

\subsection{Inference}

\begin{frame}
  \frametitle{Inference}
  Dialyzer infers the \sts\ using these steps:
  \begin{itemize}
  \item Obtain function ordering from callgraph \pause
  \item For every function:
    \begin{itemize}
    \item Assign type variables to the code variables \pause
    \item Generate constraints for the type variables \pause
    \item Solve the constraints to obtain an \st\ for the function \pause
    \end{itemize}
  \item Perform a control and dataflow analysis to eliminate dead code \pause
  \item Refine local functions from the existing calls \pause
  \item Repeat until a fixpoint is reached \pause
  \end{itemize}
  The \sts\ may be stored in a Persistent Lookup Table (PLT) to avoid
  needless recalculations or further overapproximations
\end{frame}

\begin{frame}
  \frametitle{Constraint categories}
  Constraints belong to the following structural categories: \pause
  \begin{itemize}
  \item \textbf{Simple constraints:} a certain type should be equal to
    another or subtype of another \pause
  \item \textbf{Conjunctive lists:} a list of constraints that should be
    satisfied simultaneously \pause
  \item \textbf{Disjunctive lists:} a list of constraints that
    correspond to different branches in the code \pause
  \item \textbf{Constraint references:} correspond to closures
  \end{itemize}
\end{frame}

\begin{frame}[fragile]
  \frametitle{Constraints}
\begin{code}{Sample code}
bar(1) -> 5;  % Success typing: 
bar(2) -> 10. % bar(1|2) -> 5|10

foo(a) -> b;
foo(X) -> bar(X).
\end{code}
\begin{code}{Constraints for foo}
Conjunctive List 1:
 * var(1) eq fun(var(2)) -> var(3)
 * Disjunctive List 2:
 *  * Conjunctive List 3:
 *  *  * var(2) eq a
 *  *  * var(3) eq b
 *  * Conjunctive List 4:
 *  *  * var(2) sub 1|2
 *  *  * var(4) sub 5|10
 *  *  * var(3) eq var(4)
\end{code}
\end{frame}

\begin{frame}
  \frametitle{Solving mechanics}
  \begin{itemize}
  \item A dictionary for the type variables is attached to every
    list
  \item All variables have initial value \emph{any} \pause
  \item Simple constraints tighten the variables using \emph{infimum}
    and \emph{unification} \pause
  \item Disjunctive lists' dictionaries contain the \emph{supremum} of
    the types in each list element \pause
  \item Lists are solved until they reach a fixpoint (checked using
    type \emph{equality}) \pause
  \item \textbf{Important:} The final function's type is constructed
    at top level
  \end{itemize}
\end{frame}

\section{Intersection types}

\subsection{Terminology}

\begin{frame}[fragile]
  \frametitle{Intersection types}
  \begin{itemize}
  \item Types describe sets of terms \pause
  \item An \textbf{Intersection type} describes a set whose elements
    belong simultaneously to two or more other sets \pause
  \item Overloaded functions' types can use intersections:
    \begin{code}{Simple intersection type for add}
-spec add(integer(), integer()) -> integer();
      add(float()  , float()  ) -> float().
    \end{code}
  \end{itemize}
\end{frame}

\subsection{Structure and semantics}

\begin{frame}[fragile]
  \frametitle{Structure and semantics}
  \begin{code}{Example}
-spec foo(a     ) -> 1;
      foo(b     ) -> 2;
      foo(atom()) -> 3.
  \end{code}
  \begin{itemize}
  \item List of clauses \pause
  \item Order does NOT matter \pause
  \item In a call the result is the supremum of the result of all
    applicable domains \pause
  \item Catch all clauses are included everywhere
  \end{itemize}
\end{frame}

\subsection{Inference}

\begin{frame}
  \frametitle{Using dictionaries as environments}
  \begin{itemize}
    \item We need to maintain the relations between the values of the
      type variables \pause
    \item The dictionaries attached to the lists readily provide a
      sense of environment: \pause
      \begin{itemize}
      \item \emph{Conjunctive lists} hold all the values for each code
        block \pause
      \item If every block contains a local type for the whole
        function, \emph{disjunctive lists} can find a proper
        \emph{supremum} type for all of their elements
      \end{itemize}
  \end{itemize}
\end{frame}

\begin{frame}
  \frametitle{Changes in type operators} 
  The following changes are introduced with regard to the function
  type: \pause
  \begin{itemize}
  \item \textbf{Supremum:} instead of taking the supremum of arguments
    and result, simply add the new clauses to the list \pause
  \item \textbf{Infimum:} take the infimum of every pair of clauses
    \pause
  \item \textbf{Equality:} clauses are ordered and \emph{reduced} so
    that equality checks are kept fast and efficient
  \end{itemize} 
  Other operators needed minor changes due to the change in the type
  structure
\end{frame}

\begin{frame}[fragile]
  \frametitle{Clauses reduction examples}
\begin{code}{Clauses reduction examples}
  Example 1: Same domains in supremum
  -----------------------------------
  Type A  : fun((a) -> b)
  Type B  : fun((a) -> c)
  Result  : fun((a) -> b; (a) -> c)
  Reduced : fun((a) -> b | c)
  -----------------------------------
  Example 2: Same ranges in supremum
  -----------------------------------
  Type A  : fun((a) -> c)
  Type B  : fun((b) -> c)
  Result  : fun((a) -> c; (b) -> c)
  Reduced : fun((a | b) -> c)
  -----------------------------------
  -----------------------------------
  Initial  : fun((a,c)->e;(a,d)->e;(b,c)->e;(b,d)->e)
  1st pass : fun((a,c|d) -> e; (b,c|d) -> e)
  2nd pass : fun((a|b,c|d) -> e)
\end{code}
\end{frame}

\begin{frame}
  \frametitle{Testing the new type operators}
  \textbf{Testing type operations:} \emph{Property testing} \pause
  \begin{itemize}
  \item If Inf is the infimum of F1 and F2 then Inf is a subtype of
    both of them
  \item If Sup is the supremum of F1 and F2 then F1 and F2 are
    subtypes of Sup
  \item If Sup is the supremum of F1 and F2, Inf1 is the infimum of
    Sup and F1, Inf2 is the infimum of Sup and F2 then Inf1 is equal
    to F1 and Inf2 is equal to F2
  \item The supremum/infimum of F1 and F2 is equal to the
    supremum/infimum of F2 and F1
  \item If F1 is subtype of F2 and F2 is subtype of F1 then they are
    equal
  \end{itemize} \pause
  PropEr was used for the generation of random function types and the
  testing.
\end{frame}

\begin{frame}[fragile]
  \frametitle{Towards our goal}
  \begin{itemize}
    \item \emph{Supremum} is now ready to create an intersectioned
      type if the clauses are solved completely \pause
    \item We need to ``push'' the main constraint into each clause \pause
    \item This might not be enough...
      \begin{code}{Nested branches}
foo(X, Y) ->
  case X of
   a -> b;
   b -> case Y of
         a -> c;
         b -> d
        end
  end.
      \end{code}
  \end{itemize}
\end{frame}

\begin{frame}
  \frametitle{Disjunctive normal form}
  \begin{itemize}
  \item The \textbf{disjunctive normal form} of the constraints can be
    used to:
    \begin{itemize}
      \item ``push'' the main constraint into each clause
      \item simplify nested branches
      \item fallback whenever the constraints are complex
    \end{itemize} \pause
  \item We also need to generate disjunctive list constraints when
    calls are present in the code (\emph{this solves the situation in
      behaviours})
    \begin{itemize}
    \item easy for already solved functions
    \item more complicated for self-calls and members of a strongly
      connected component (SCC)
    \end{itemize}
  \end{itemize}
\end{frame}

\begin{frame}
  \frametitle{Self recursive functions and SCCs}
  \begin{itemize}
  \item Dialyzer handles self recursive and SCC calls in a special
    way: \pause
    \begin{itemize}
    \item Initially these calls are assumed to fail and are excluded
      from the success typing \pause
    \item After the first calculation the base cases have been found
      and are used as the result of the calls \pause
    \item This is repeated until a fixpoint is reached \pause
    \end{itemize}
  \item Constraint generation and solving are separate procedures
    therefore \emph{dynamic constraints} need to be added for these
    calls \pause
  \item These are substituted in every iteration with the appropriate
    \emph{disjunctive constraint}
  \end{itemize}
\end{frame}

\begin{frame}
  \frametitle{We are ready!}
  In short: \pause
  \begin{itemize}
  \item During constraint generation constraints from calls become
    either \emph{disjunction lists} or \emph{dynamic constraints} \pause
  \item Prior to solving obtain the \emph{disjunctive normal form} of
    the constraints (for Self-recursive functions and SCCs this is
    done in every iteration, after \emph{dynamic constraints} have
    been substituted) \pause
  \item Solve and get the intersectioned type \pause
  \item ... or an old-style type if calculation of normal form is
    deemed to be inefficient
  \end{itemize}
\end{frame}

\begin{frame}
  \frametitle{One more refinement...}
  \begin{itemize}
  \item Dialyzer's refinement works by collecting the
    \emph{supremum} of all calls to a local function and
    using it to restrain the domain \pause
  \item We can (and should) keep them separate \pause
  \item Thus we practically refine \emph{each} separate
    call (\emph{polyvariant control flow analysis})
  \end{itemize}
\end{frame}

\begin{frame}[fragile]
  \frametitle{Performance issues}
  In the following cases we fall back to the old single clause type
  \begin{itemize}
    \item Too deep \emph{conjunctive normal form}
    \item SCCs with many functions (e.g. automatically generated
      parsers)
    \item Self recursive functions or SCCs that fail to reach fixpoint
      after several tries:
\begin{code}{A self recursive numeric identity function}
    id(0) -> 0; id(N) -> 1 + id(0)
\end{code}
  \end{itemize}
\end{frame}

\begin{frame}
  \frametitle{Detecting discrepancies}
  \begin{itemize}
    \item The actual usage of the success typings for discrepancy
      detection happens in a final dataflow pass on the code under
      inspection
    \item The only change is the substitution of the generic lookup
      for the return type of function calls with a lookup that takes
      into consideration the types of the arguments
  \end{itemize}
\end{frame}

\section{Application: Analysis of behaviours}

\subsection{What are behaviours?}

\begin{frame}
  \frametitle{Behaviours}
  \begin{itemize}
  \item \textbf{Behaviours:} the equivalent of object-oriented
    languages' \emph{abstract classes} or \emph{interfaces} \pause
  \item They divide the functionality of a component into a generic
    part (the \emph{behaviour} module) and a specific part (the
    \emph{callback} module) \pause
  \item OTP includes behaviours for:
    \begin{itemize}
    \item applications
    \item servers
    \item finite state machines
    \item event handlers
    \item ...
    \end{itemize}
  \end{itemize}
\end{frame}

\begin{frame}
  \frametitle{OTP Behaviours}
  \begin{fulltable}{|c|c|c|}{Common OTP behaviours}{tab:otp_behaviours}
\hline
Module & Description & Callbacks \\
\hline
\hline
\multirow{4}{*}{gen\_server.erl}
& Generic server behaviour. Contains a state  & init,        \\
& that is manipulated by calls (that require  & handle\_call,\\
& a reply) and casts (that do not wait for a  & handle\_cast,\\
& reply).                                     & terminate    \\
\hline
\multirow{5}{*}{gen\_fsm.erl}
& Finite state machine behaviour. A finite    & init, State1,\\
& number of states exist along with the       & State2, ..., \\
& messages each state accepts, the replies    & StateN,      \\
& that are sent and the state change that     & terminate    \\
& may follow.                                 &              \\ 
\hline
\multirow{3}{*}{gen\_event.erl}
& Generic event handler. Event handlers       & init,         \\
& register in a central event manager and     & handle\_event,\\
& are notified for any event that arrives.    & terminate     \\
\hline
\multirow{4}{*}{application.erl}
& \er\ application. An application is a       &       \\
& collection of modules that implement some   & start,\\
& specific functionality and can be started   & stop  \\
& and stopped as a whole.                     &       \\
\hline
\multirow{4}{*}{supervisor.erl}
& A process which supervises other processes  &     \\
& called child processes. A child process can & init\\
& either be another supervisor or a worker    &     \\
& process.                                    &     \\
\hline
\end{fulltable}

\end{frame}

\subsection{Definition of a behaviour}

\begin{frame}[fragile]
  \frametitle{How did behaviours look like?}
\begin{code}{Generic server's existing declaration of callbacks}
-export([behaviour_info/1]).

behaviour_info(callbacks) ->
  [{init,1}, {handle_call,3}, {handle_cast,2},
   {terminate, 2}, {code_change, 3}].

%%% The user module should export:
%%%
%%%   init(Args)
%%%     ==> {ok, State}
%%%         {ok, State, Timeout}
%%%         ignore
%%%         {stop, Reason}
%%%
%%% ....  COMMENTS FOR THE OTHER CALLBACKS HERE .....
\end{code}
\end{frame}

\begin{frame}[fragile]
  \frametitle{New declaration of behaviours}
\begin{code}{Declaration of the same callbacks using attributes}
-callback init(Args :: term()) ->
    {ok, State :: term()} |
    {ok, State :: term(), timeout() | hibernate} |
    {stop, Reason :: term()} | 
    ignore.

%%% .... MORE CALLBACK ATTRIBUTES .... %%%
\end{code}
\end{frame}

\subsection{Checking the use of a behaviour}

\begin{frame}
  \frametitle{Checking the use of a behaviour}
  \begin{itemize}
    \item Dialyzer can infer a \emph{success typing} for the callbacks
      \pause
    \item Using the \emph{-callback} attributes as reference we can
      ensure that the callback functions:
      \begin{itemize}
        \item accept the required arguments \pause
        \item return values that belong to the expected results
      \end{itemize}
  \end{itemize}
\end{frame}

\begin{frame}[fragile]
  \frametitle{A simple discrepancy}
\begin{code}{Gen\_server's handle cast callback}
-callback handle_cast(Request :: term(),
                      State :: term()) ->
  {noreply, NewState :: term()} | 
  {noreply, NewState :: term(), timeout() | hibernate} | 
  {stop, Reason :: term(), NewState :: term()}.
\end{code}
\begin{code}{Example from httpd\_request\_handler.erl}
handle_cast(Msg, State) ->
    {reply, {cast_api_violation, Msg}, State}.
\end{code}
\end{frame}

\subsection{False positive warnings}

\begin{frame}[fragile]
  \frametitle{A false warning}
  Some cases are not straightforward...
\begin{code}{Simplified example from mnesia\_recover.erl}
  handle_info({connect_nodes, Ns, From}, State) ->
    handle_call({connect_nodes,Ns},From,State);

  handle_call({disconnect, Node}, _From, State) ->
    ...
    {reply, ok, State};
  handle_call({connect_nodes, Ns}, From, State) ->
    ...
    {noreply, State}.
\end{code}
  \pause
  \begin{itemize}
  \item A warning is emitted for handle\_info, even though the call is
    safe. \pause
  \item Using intersection types we get the correct, limited type!
  \end{itemize}
\end{frame}

\section{Results}

\begin{frame}
  \frametitle{Results from behaviours}
  \begin{fulltable}{|c|c|c|c|}{Behaviour Discrepancies in OTP Applications}{tab:erl_apps_behaviours}
\hline
Application & Description & Behaviour Used & Discrepancies\\
\hline
\hline
\multirow{2}{*}{inets}& \multirow{2}{*}{Internet clients and servers}&
gen\_server & 1 \\
\cline{3-4}  
& & tftp & 1 \\
\hline
dist\_ac & distributed application controller & gen\_server & 1 \\
\hline
mnesia & distributed DBMS & gen\_server & 2 \\
\hline
ssh & SSH application & gen\_server & 2 \\
\hline
error\_logger & Stdlib's error logger & gen\_server & 1 \\
\hline
\hline
\multicolumn{3}{|c|}{\textbf{Total discrepancies}} & 8 \\
\hline
\end{fulltable}

\end{frame}

\begin{frame}
  \frametitle{Results from intersections}
  \begin{fulltable}{|c|c|c|c|}{New Discrepancies in OTP Applications}{tab:intersection_results}
\hline
Application & Description & Category & Discrepancies\\
\hline
\hline
asn1 & Abstract Syntax Notation 1 tools & Exit call & 1 \\
\hline
auth & Network Authentication Server & Deriving warnings & 9 \\
\hline
edoc & Documentation generator & Failing call & 1 \\
\hline
\multirow{2}{*}{erts} & \multirow{2}{*}{Erlang Run-Time System} &
Failing calls & 2 \\
\cline{3-4}
& & Deriving warnings & 17 \\
\hline
file & File Interface Module & Deriving warnings & 25 \\
\hline
hipe & High Performance compiler & Unneeded case & 1 \\
\hline
\multirow{3}{*}{inets}& \multirow{3}{*}{Internet clients and servers} &
Failing calls & 2 \\
\cline{3-4}
& & Exit call & 1 \\
\cline{3-4}
& & Nonmatching clause & 1 \\
\hline
\multirow{2}{*}{mnesia} & \multirow{2}{*}{distributed DBMS}
 & Exit calls & 7 \\
\cline{3-4}
& & Unneeded case & 1 \\
\hline
ssh & SSH application & Unneeded case & 1 \\
\hline
\multirow{3}{*}{ssl} & \multirow{3}{*}{Interface for Secure Socket Layer} &
Failing calls & 2 \\
\cline{3-4}
 & & Unneeded case & 1 \\
\cline{3-4}
& & Deriving warnings & 1 \\
\hline
\multirow{2}{*}{tv} & graphical examination of & \multirow{2}{*}{Unneeded case} & \multirow{2}{*}{1} \\
& ETS and Mnesia tables & & \\
\hline
\hline
\multicolumn{3}{|c|}{\textbf{Total original errors}} & 22 \\
\hline
\end{fulltable}

\end{frame}

\begin{frame}[fragile]
  \frametitle{Examples for intersections}
  \begin{code}{A call that will surely fail}
httpd_request_handler.erl:439: The call

  httpd_response:send_status(ModData, 501, [1..255,...])

  will never return since it differs in the 2nd and/or
  3rd argument from the success typing arguments:

  (#mod{}, 100|304|400|408|413|416|500|503, any()) or
  (#mod{}, 301|403|404|414, [any()]) or
  (#mod{}, 400|401|412, 'none') or
  (#mod{}, 501, {atom() | [any()],[any()],[any()]})

  %% A rather esoteric warning about the 501
  %% Not Implemented HTTP status message.
\end{code}
\end{frame}

\begin{frame}[fragile]
  \frametitle{Examples for intersections}
\begin{code}{A redundant catch-all clause}
con_desc(E) ->
  case cerl:type(E) of
    cons -> {?cons_id, 2};
    tuple -> {?tuple_id, cerl:tuple_arity(E)};
    binary -> {?binary_id, cerl:binary_segments(E)};
    literal -> case cerl:concrete(E) of
                 [_|_] -> {?cons_id, 2};
                 T when is_tuple(T) ->
                   {?tuple_id, tuple_size(T)};
                 V -> {?literal_id(V), 0}
               end;
    _ ->
      throw({bad_constructor, E})
  end.

cerl_pmatch.erl:338(13): The variable _ can never match
since previous clauses completely covered the type
'binary' | 'cons' | 'literal' | 'tuple'
\end{code}

\end{frame}

\begin{frame}[fragile]
  \frametitle{Examples for intersections}
\begin{code}{A path that ends in a call to \emph{exit}}
check_if_valid_tag(<<>>, _, OptOrMand) ->
    check_if_valid_tag2(false,[],[],OptOrMand);

check_if_valid_tag2(_Class_TagNo, [], Tag, MandOrOpt) ->
    check_if_valid_tag2_error(Tag,MandOrOpt);

check_if_valid_tag2_error(Tag,mandatory) ->
    exit({error,{asn1,{invalid_tag,Tag}}});
check_if_valid_tag2_error(Tag,_) ->
    exit({error,{asn1,{no_optional_tag,Tag}}}).

asn1rt_ber_bin.erl:2200(2): The call
 asn1rt_ber_bin:check_if_valid_tag2('false',[],[],any())
 will never return since it differs in the 2nd argument
 from the success typing arguments: ('false' |
 {'APPLICATION',_} | {'CONTEXT',_} | {'PRIVATE',_} |
 {'UNIVERSAL',_},nonempty_maybe_improper_list(),[] |
 {_,_,_},any())
\end{code}
\end{frame}

\begin{frame}
  \frametitle{Demo!}
  \begin{center}
    Demo!
  \end{center}
\end{frame}

\section{Conclusion}

\begin{frame}
  \frametitle{Related work}
  The general topic of static type analysis for languages with dynamic
  type systems is not new. There have been approaches similar to
  Dialyzer for:
  \begin{itemize}
  \item Ruby (Diamondback Ruby)
  \item Dylan (which is based on Lisp)
  \item JavaScript
  \end{itemize}
\end{frame}

\begin{frame}[fragile]
  \frametitle{Further work}
  \begin{itemize}
  \item Negative types for Dialyzer
    \begin{itemize}
    \item Completes the semantics of the language
    \item Will help in catching more discrepancies:
\begin{code}{A discrepancy we still miss}
foo(a) -> b;
foo(X) -> X.

bar() ->
    a = foo(a).
\end{code}
    \end{itemize} \pause
  \item Tighter coupling between the type and the code \pause
  \item Advanced refinement using ``covers''
  \end{itemize}
\end{frame}

\begin{frame}
  \frametitle{Bibliography}
  \bibliographystyle{abbrv}
  \bibliography{ref_presentation}
\end{frame}

\begin{frame}
  \begin{center}
    Thank you!
  \end{center}
\end{frame}

\end{document}

% xelatex -interaction=nonstopmode presentation_v2.tex; evince presentation_v2.pdf
