% DOCUMENT FORMAT ==============================================================

\documentclass{ntua-thesis} % a4paper,11pt,twoside,titlepage already set
% \pagestyle{plain} % pagestyle already set
% \usepackage[margin=2.5cm]{geometry} % margins already set


% PACKAGE SETTINGS =============================================================

\usepackage[cm-default]{fontspec}
\usepackage{amsmath}
\usepackage{amsfonts}
\usepackage{multirow}
\usepackage{array}
\usepackage{mdwlist}
\usepackage{graphicx}
\usepackage{xunicode}
\usepackage{xltxtra}
\usepackage{url}
\usepackage[colorlinks]{hyperref}
\hypersetup{
   bookmarksnumbered,
   citecolor={magenta},
   linkcolor={blue},
   urlcolor={red},
   pdfpagemode={UseOutlines}
}
\setromanfont[Mapping=tex-text]{CMU Serif}
\setsansfont[Mapping=tex-text]{CMU Sans Serif}
\setmonofont[Mapping=tex-text]{CMU Typewriter Text}
\setmainfont[Mapping=tex-text]{CMU Serif}


% CUSTOM COMMANDS ==============================================================

\newenvironment{fulltable}[3]{
    \def\tempcaption{#2}
    \def\templabel{#3}
    \begin{table}[hbtp]
    \begin{center}
    \begin{tabular}[c]{#1}
}{
    \end{tabular}
    \end{center}
    \caption{\tempcaption\label{\templabel}}
    \end{table}
}


% CODE HIGHLIGHTING ============================================================

\usepackage{listings}
\lstdefinelanguage{erlang}{
    morekeywords={
	% Reserved words
	after,and,andalso,band,begin,bnot,bor,bsl,bsr,bxor,case,catch,cond,div,
	end,fun,if,let,not,of,or,orelse,query,receive,rem,try,when,xor,
	% Common attributes
	behavior,behaviour,callback,compile,export,export_type,import,include,
	include_lib,module,opaque,record,spec,type,
	% Macro-related attributes
	define,ifdef,ifndef,else,endif,undef
    },
    morekeywords=[2]{
	% Process-related BIFs
	apply,exit,throw,get,put,erase,%error,
	% Useful general purpose BIFs
	%abs,min,max,trunc,round,size,bit_size,byte_size,tuple_size,element,
	%setelement,length,hd,tl,
	% Type test BIFs
	is_atom,is_binary,is_bitstring,is_boolean,is_float,is_function,
	is_integer,is_list,is_number,is_pid,is_port,is_record,is_reference,
	is_tuple,
	% Type conversion BIFs
	atom_to_binary,atom_to_list,binary_to_atom,binary_to_list,
	binary_to_term,bitstring_to_list,integer_to_list,list_to_atom,
	list_to_binary,list_to_bitstring,list_to_float,list_to_integer,
	list_to_pid,list_to_tuple,term_to_binary,tuple_to_list,float_to_list,
	pid_to_list
    },
    morekeywords=[3]{
	% Type names
	integer,non_neg_integer,pos_integer,neg_integer,float,atom,binary,
	bitstring,pid,port,reference,function,tuple,list,any,none,
	% Common atoms
	true,false,ok,error,undefined
    },
    otherkeywords={->,!},
    morecomment=[l]\%,
    morestring=[b]",
    morestring=[b]'
}[keywords,comments,strings]
\lstset{
    numbers=left,
    numberstyle=\tiny,
    basicstyle=\ttfamily\footnotesize,
    basewidth=0.59em,
    keywordstyle=[3]{},
    commentstyle=\itshape\footnotesize,
    tabsize=8,
    frame=single,
    frameround=tttt,
    showstringspaces=false,
    breaklines=false,
    captionpos=b,
    aboveskip=\bigskipamount,
    belowskip=\bigskipamount,
    escapechar=^
}
% Style options:
% numberstyle,basicstyle,identifierstyle,commentstyle,stringstyle
% keywordstyle=[1]{},keywordstyle=[2]{},directivestyle
% \small\tiny\footnotesize\itshape\ttfamily\bf

\lstnewenvironment{code}[2]{
    \nopagebreak
    \lstset{language=erlang,float=htbp,label={#1},caption={#2}}
}{}
\lstnewenvironment{console}[2]{
    \nopagebreak
    \lstset{float=htbp,label={#1},caption={#2},numbers=none}
}{}

\title{Τεχνικές βελτίωσης της αποτελεσματικότητας εύρεσης λαθών
       σε προγράμματα μέσω στατικής ανάλυσης}
\author{Σταύρος Αρώνης}
\thesis[του]{Σταύρου Αρώνη}
\presenting{8}{1}{2011}
\supervisor[Αν. Καθηγητής ]{Κωστής Σαγώνας}
\epitropiF[Επικ. Καθηγητής ]{Νικόλαος Παπασπύρου}
\epitropiS[Αν. Καθηγητής ]{Κώστας Κοντογιάννης}

\department{Σχολή Ηλεκτρολόγων Μηχανικών και Μηχανικών Υπολογιστών}
\division{Τομέας Τεχνολογίας Πληροφορικής και Υπολογιστών}
\lab{Εργαστήριο Τεχνολογίας Λογισμικού}

%% Custom commands.
\newcommand{\er}{\mbox{\textsc{Erlang}}}
\newcommand{\dr}{\mbox{\textsc{Dialyzer}}}
\newcommand{\any}{\emph{any}}
\newcommand{\none}{\emph{none}}
\newcommand{\atom}{\emph{atom}}
\newcommand{\atoms}{\emph{atoms}}
\newcommand{\genserv}{\emph{gen\_server}}
\begin{document}

% \frontmatter
\maketitle
\def\templen{\parindent}
\setlength{\parindent}{0pt}
\setlength{\parskip}{1.5ex plus 0.5ex minus 0.2ex}
\begin{abstract}

\hyphenation{α-ντι-κει-με-νο-στρε-φών εν-το-πι-σμό}

Η ανίχνευση και διόρθωση λαθών σε προγράμματα είναι μια διαδικασία που
καταναλώνει σημαντικό μέρος του χρόνου κάθε προγραμματιστή. Εργαλεία που
διευκολύνουν τον εντοπισμό λαθών είναι χρήσιμα τόσο στον περιορισμό των τελικών
λαθών, όσο και στην διαδικασία εντοπισμού και διόρθωσης κατά την ανάπτυξη του
προγράμματος. Στην παρουσα εργασία παρουσιάζεται η επέκταση των δυνατότητων του
\dr, ενός προγράμματος που χρησιμοποιεί στατική ανάλυση για την ανίχνευση λαθών 
σε προγράμματα στη γλώσσα \er, με την εισαγωγή \emph{τύπων τομής} που βελτιώνουν
σημαντικά την ακρίβεια του εργαλείου και την προσθήκη επιπλέον δυνατοτήτων σχετικών
με τον εντοπισμό λαθών στην χρήση των \emph{behaviours}, που αντιστοιχούν στις
\emph{αφηρημένες κλάσεις} των αντικειμενοστρεφών γλωσσών προγραμματισμού. Οι 
επεκτάσεις αυτές οδήγησαν στον εντοπισμό σημαντικών λαθών σε ήδη υπάρχοντα και
ενδελεχώς ελεγμένο κώδικα.

\begin{keywords}
Στατική ανάλυση, Συμπερασμός τύπων, Τύποι τομής, Αφηρημένες κλάσεις, \er, \dr
\end{keywords}
\end{abstract}

\begin{abstracteng}

Detection and correction of bugs consumes a significant amount of every
developer's time. Any tool designed with the intention to make this task easier
is useful to both minimize the final bugs present in the code and to detect and
correct them before release. This thesis describes the extension of \dr, a static 
analysis tool designed to find discrepancies in \er\ programs, with the
introduction of \emph{intersection types} that lead to impovements in its accuracy 
and the addition of a new module capable of finding errors in the use of
\emph{behaviours}, which correspond to \emph{abstract classes} as they appear in
object-oriented languages. These extensions lead to the detection of important
errors in already thoroughly tested code.

\begin{keywordseng}
Static analysis, Type inference, Intersection types, Behaviours, Abstract classes, \er, \dr
\end{keywordseng}
\end{abstracteng}

\setlength{\parindent}{\templen}
\setlength{\parskip}{0pt}
\tableofcontents
% \listoffigures
\listoftables
\renewcommand{\lstlistlistingname}{List of Listings}
\lstlistoflistings % changed the title above

% \mainmatter
% moved these two commands here so that they don't influence the toc
\setlength{\parindent}{0pt}
\setlength{\parskip}{1.5ex plus 0.5ex minus 0.2ex}

\chapter{Introduction}

\dr\ is one of the most widely-used tools in the development of
\er\ programs. Its name stands for DIscrepancy anALYZer for ERlang and
it does exactly that: in an inherently dynamically typed language such
as \er, \dr\ is able to detect many type related discrepancies using
static analysis. The initial version reported type errors using
success typings but subsequent extensions allowed for verification of
user contracts, detection of violations of the opaqueness of certain
abstract data types and recently even warnings about race conditions
\cite{Races@PADL-10, springerlink:10.1007/978-3-540-30477-7_7,
  SuccessTypings@PPDP-06, opaques}.

Altough taking part in the development of such a tool is a pleasure in
itself, this thesis begun with a more concrete motivation: the
extension of \dr\ to detect discrepancies in the use of \er's
\emph{behaviours} (more on these in
Section \ref{sct:behaviour_preliminaries}), including simple
type-related checks and enabling the new race detection analysis to
pass through them undistracted. This goal was accomplished easily, as
described in Chapter \ref{chp:behaviours}, but the cost was that the
soundness of \dr\ was compromised (see Section
\ref{sct:behaviour_discrepancies}).

As this couldn't be tolerated (one of \dr's cornerstones is that it'll
never be wrong about a warning it emits) the second part of this
thesis came into focus. When you design a tool that promises to detect
discrepancies in code soundly the main question you have to address is
not what to include but what to leave outside. One of the greatest
compromises made in \dr's initial design was that it would work with a
type system \textbf{without} \emph{intersection types} for functions
(see Section \ref{sct:dialyzer_preliminaries}). This lead to its
inability to report glaring errors as the one presented in Section
\ref{sct:intersection_preliminaries} and was the reason behind the
violation of soundness in behaviour analysis as well.

As in every happy story, this is no longer the case! Chapters
\ref{chp:intersection_generate} and \ref{chp:intersection_usage} show
how intersection types can be generated and used. \dr\ emerges
stronger than before, able to catch both abuses of behaviours and a
whole new range of actual errors in code. What is more, further
improvements are now easily attainable and are presented in Section
\ref{sct:further_work}.

\chapter{Preliminaries}

\section{\er\ and OTP}
\label{sct:erlang_preliminaries}

\er\ is a strict, dynamically typed functional programming language
with support for concurrency, communication, distribution,
fault-tolerance, on-the-fly code reloading, automatic memory
management and support for multiple platforms \cite{joe_book1}. The
number of areas where \er\ is actively used is increasing. However,
its primary application area is still in large-scale embedded control
systems developed by the telecom industry.

The main implementation of the language, the \er/OTP (Open Telecom
Platform) system from Ericsson, has been open source since 1998 and
has been used quite successfully both by Ericsson and by other
companies around the world to develop software for large commercial
applications.

Nowadays, applications written in the language are significant both in
number and in code size making \er\ one of the most industrially
relevant declarative languages.

\section{Behaviours}
\label{sct:behaviour_preliminaries}

As expected from an industrially used framework, OTP provides
components that make the use of the language easier. Using these
components developers are able to take advantage of the aforementioned
language features. An excellent guide on how to develop a
fault-tolerant, distributed and concurrent application is available to
help new developers learn the language quickly \cite{des_princ}.

One of the key elements of the framework are the
\emph{behaviours}. These correspond to the \emph{abstract classes} or
\emph{interfaces} found in object-oriented languages, like Java, as
they divide the functionality of a component into a generic part (the
\emph{behaviour} module) and a specific part (the \emph{callback}
module).

The behaviour module is part of Erlang/OTP. To implement a process
such as a server, the user only has to implement the callback module
which should export a pre-defined set of functions, the callback
functions. It is also possible for a developer to design his own
behaviour, either by splitting the functionality of the program or by
following a known design pattern\cite{behaviours}.

Discrepancies related with \emph{behaviours} are presented in Chapter
\ref{chp:behaviours}.

\section{\dr}
\label{sct:dialyzer_preliminaries}

\dr\ is a static analysis tool included in the OTP since 2007. It can
detect a wide variety of discrepancies (i.e., type errors, software
defects such as exception-raising code, hidden failures, unsatisfiable
conditions, redundancies such as unreachable code, race conditions,
etc.) in single modules or entire applications. Dialyzer is totally
automatic, extremely easy to use and particularly successful in
identifying software defects which may be hidden in Erlang code,
especially in program paths which are not exercised by testing.

In the heart of \dr\ lies a soft typing system. Its purpose is
essentially to capture the biggest set of terms for which it can be
proven that type clashes will occur. The type signatures that
\dr\ infers, called \emph{success typings}, are the complement of that
set of terms. Success typings are an over-approximation to the set of
terms for which a function can evaluate: the domain of the signature
includes all possible values that the function could accept as
parameters, and its range includes all possible return values for this
domain. Success typings are guaranteed to capture all intended uses of
a function, along, perhaps, with some erroneous ones. Thus, any use of
a function that is incompatible with its success typing will
definitely fail. In effect, success typings approach the type
inference problem from a direction opposite to that of type systems
for statically typed languages.

For the actual workings of \dr\ more details can be found in the
relevant bibliography \cite{Elli, type_system,
  springerlink:10.1007/978-3-540-30477-7_7, SuccessTypings@PPDP-06,
  GradualTyping@Erlang-08}. Only some key features will be presented
concisely as they are relevant with the modifications this thesis
describes.

\subsection{Analysis phases}

\dr\ operation can be split in two phases:

\begin{enumerate}
\item \textbf{Find success typings:} In this phase \dr\ traverses the
  code of every function included in the analysis and finds the
  success typing of each function. This requires several iterations of
  simple constraint solving and dataflow analysis. In the end, every
  function is assigned a final success typing.
\item \textbf{Emit warnings:} After the success typings have been
  fixed the code is traversed one more time. In this run a warning is
  emitted whenever a discrepancy is detected.
\end{enumerate}

\subsection{Refinement of success typings}

In the calculation of success typings, function's that are not
exported undergo a further refinement. The actual calls to these
functions are used to calculate the effective domain. This is
possible, as all the calls are located in the module under
analysis. Restricting the domain may tighten the final success typing
and possibly render some clauses unneeded. In Listing \ref{lst:refine}
an example is provided. Even though $foo$'s initial type will be
calculated as $number \rightarrow number$ the final success typing
will be $42 \rightarrow 43$ allowing a stricter return type for $test$
as well (initially $number$, finally $43$).

\begin{console}{lst:refine}{Refining a local function}
-module(refine).
-export([test/0]).

test() ->
  foo(42).

foo(X) -> X + 1;
\end{console}

\subsection{Contracts}

\dr\ can take into account annotations placed by developers to further
restrict the success typings. These are called $specs$ and are used
for both type checking (by \dr) and documentation (by
\mbox{\textsc{Edoc}})\cite{type_system}. Their syntax is presented in
\cite{type_system} and examples are given in Listing
\ref{lst:specs}. The clauses of a spec should not overlap (a certain
call must belong in exactly one of them).

\begin{console}{lst:specs}{Valid and invalid specs}
%% Valid spec with one clause
-spec foo(number()) -> number().

%% Valid spec with two clauses
-spec foo(integer()) -> integer();
         (  float()) -> float().

%% Invalid spec. 'bar' is an atom and overlaps with the 2nd clause
-spec foo( 'bar') -> 'ok';
         (atom()) -> 'error'.
\end{console}

\section{Intersection types}
\label{sct:intersection_preliminaries}

Intersection types are types describing values that belong to both of
two other given types. For example, in most implementations of C the
signed char has range -128 to 127 and the unsigned char has range 0 to
255, so the intersection type of these two types would have range 0 to
127. Such an intersection type could be safely passed into functions
expecting either signed or unsigned chars, because it is compatible
with both types.

Most modern statically typed programming languages support overloaded
functions. Such functions execute different code depending on the type
of the arguments they receive. Intersection types are useful for
describing the type of such functions: an example is a function with
type $ Int \rightarrow Int | Float \rightarrow Float $. This function
will return $Int$ if called with an $Int$ argument and a $Float$ if
called with $Float$. Type checking is an essential part of the
semantic analysis of a statically typed language's compiler.

Unfortunately, \er\ is a dynamic typed language, so type checking is
not part of the compilation. What is more, the relevant tool, \dr, was
designed without intersection types. In Listing
\ref{lst:trivial_intersection} we see a simple overloaded function,
$foo$. \dr\ using \emph{success typings} will find that the function
succeeds for arguments $a$ or $b$ and the return may be $1$ or
$2$. Lacking intersection types, this will be expressed as $a | b
\rightarrow 1 | 2 $. This is an overapproximation because $foo$ will
never return 2 if the argument is $a$ and vice-versa. This will cause
the error presented in the combination of $foo$ with the other
functions in the example to pass undetected \footnote{In fact this
  will happen only if $buz$ is exported. Otherwise the refinement
  described in \ref{sct:dialyzer_preliminaries} will find that buz
  will be called with $b$ only and will refine $foo$'s type with this
  info, catching the error.}

\begin{console}{lst:trivial_intersection}{Trivial intersection-related error}
  foo(a) -> 1;
  foo(b) -> 2.

  bar(1) -> ok.

  buz(X) ->
    bar(foo(X)).

  test() ->
    buz(b).
\end{console}

The goal of the second part of this thesis is the inference of
intersection types and is presented in Chapters
\ref{chp:intersection_generate} and \ref{chp:intersection_usage}.

\chapter{Finding discrepancies in behaviour usage}
\label{chp:behaviours}

As described in Section~\ref{sct:behaviour_preliminaries}, behaviours
are \er's equivalent of \emph{abstract classes}, as they appear in
object-oriented languages like Java. When using \emph{abstract
  classes} the developer might make trivial mistakes such as
forgetting to implement a particular abstract method or implementing
it incorrectly so that it ``doesn't fit'' with those already
provided. In \er\ the abstract methods are called
\emph{callbacks}. \er's compiler detects only the lack of
implementation of any \emph{callbacks} but \dr\ can be used to further
aid the developer by verifying whether his implementations have the
expected success typings, ensuring thus that they ``fit well'' with
the already provided infrastructure. \dr's recent feature, race
analysis~\cite{Races@PADL-10}, can also be extended to detect races
present in code that uses behaviours. Each extension will be presented
in a separate section.

\section{Usage of behaviours}

\er\ developers use behaviours heavily as they readily provide some of
the key features of the language (concurrency, communication,
distribution and fault-tolerance) and allow the developer to focus on
the particular aspects of his implementation, ignoring these
parameters. In Table~\ref{tab:otp_behaviours} the most common
behaviours in OTP are presented, along with some of the callbacks they
require.

\begin{fulltable}{|c|c|c|}{Common OTP behaviours}{tab:otp_behaviours}
\hline
Module & Description & Callbacks \\
\hline
\hline
\multirow{4}{*}{gen\_server.erl}
& Generic server behaviour. Contains a state  & init,        \\
& that is manipulated by calls (that require  & handle\_call,\\
& a reply) and casts (that do not wait for a  & handle\_cast,\\
& reply).                                     & terminate    \\
\hline
\multirow{5}{*}{gen\_fsm.erl}
& Finite state machine behaviour. A finite    & init, State1,\\
& number of states exist along with the       & State2, ..., \\
& messages each state accepts, the replies    & StateN,      \\
& that are sent and the state change that     & terminate    \\
& may follow.                                 &              \\ 
\hline
\multirow{3}{*}{gen\_event.erl}
& Generic event handler. Event handlers       & init,         \\
& register in a central event manager and     & handle\_event,\\
& are notified for any event that arrives.    & terminate     \\
\hline
\multirow{4}{*}{application.erl}
& \er\ application. An application is a       &       \\
& collection of modules that implement some   & start,\\
& specific functionality and can be started   & stop  \\
& and stopped as a whole.                     &       \\
\hline
\multirow{4}{*}{supervisor.erl}
& A process which supervises other processes  &     \\
& called child processes. A child process can & init\\
& either be another supervisor or a worker    &     \\
& process.                                    &     \\
\hline
\end{fulltable}


Developers may also write their own behaviours, whenever a common
infrastructure may be used for many specific implementations.

\subsection{Declaration of a behaviour}

A module describing a behaviour exports a specific function:
\texttt{behaviour\_info(callbacks)}. This returns the expected
callbacks in the form of a list of tuples containing the names of the
callback functions as atoms and their arity as integers. The example
in Listing~\ref{lst:behaviour_info} is taken from the
\genserv\ behaviour.

\begin{console}{lst:behaviour_info}{Generic server's declaration of callbacks}
-export([behaviour_info/1]).

behaviour_info(callbacks) ->
  [{init,1}, {handle_call,3}, {handle_cast,2},
   {terminate, 2}, {code_change, 3}].

%%% The user module should export:
%%%
%%%   init(Args)
%%%     ==> {ok, State}
%%%         {ok, State, Timeout}
%%%         ignore
%%%         {stop, Reason}
%%%
%%%   handle_call(Msg, {From, Tag}, State)
%%%
%%%    ==> {reply, Reply, State}
%%%        {reply, Reply, State, Timeout}
%%%        {noreply, State}
%%%        {noreply, State, Timeout}
%%%        {stop, Reason, Reply, State}
%%%              Reason = normal | shutdown | Term
%%%
%%% ....  MORE COMMENTS FOR THE OTHER THREE CALLBACKS HERE .....

\end{console}

\subsection{Better declaration of a behaviour}

Often, though not always, the behaviour module also contains some
additional information in the form of comments, as shown in
Listing~\ref{lst:behaviour_info}. The problem with comments is that
they are in free text form, often lacking some information as in the
case above, and cannot be trusted or mechanically processed.

Instead of the form described in the previous Listing, the
\texttt{behaviour\_info(callbacks)} clause can be substituted with the
attributes shown in Listing~\ref{lst:callback_attributes} which also
specify the types which are expected from these callbacks.

\begin{console}{lst:callback_attributes}{Callback attributes}
-callback init(Args :: term()) ->
    {ok, State :: term()} |
    {ok, State :: term(), timeout() | hibernate} |
    {stop, Reason :: term()} | 
    ignore.

-callback handle_call(Request :: term(), From :: {pid(), Tag :: term()},
                      State :: term()) ->
    {reply, Reply :: term(), State :: term()} |
    {reply, Reply :: term(), State :: term(), timeout() | hibernate} |
    {noreply, State :: term()} |
    {noreply, State :: term(), timeout() | hibernate} |
    {stop, Reason :: normal | shutdown | term(), Reply :: term(), 
                                                 State :: term()} |
    {stop, Reason :: term(), State :: term()}.

%%% .... MORE CALLBACK ATTRIBUTES .... %%%

\end{console}

These attributes are identical with specs so \dr\ can use these as a
reference to compare the inferred types of the callbacks.
Incidentally, the above example shows various interesting things:

\begin{enumerate}
\item Using the language of types and specs, one can provide
  information both for documentation purposes and for types as e.g. in
  \texttt{From :: {pid(), Tag :: term()}}
\item Comments are often incomplete or can easily become obsolete as
  e.g. the \texttt{hibernate} value is nowhere mentioned.
\end{enumerate}

\section{Finding discrepancies in callbacks}
\label{sct:behaviour_discrepancies}

\dr's extension to use these callback attributes was simple and
straightforward. After the success typings were calculated, they were
compared against the attributes and warnings were emitted when the
latter were not subtypes of the former. In this way the discrepancies
shown in Table~\ref{tab:erl_apps_behaviours} were found \footnote{Only
  definite results are presented. Some more results need verification
  from the OTP team as the documentation on which the callback
  attributes were based might be outdated.}

\begin{fulltable}{|c|c|c|c|}{Behaviour Discrepancies in OTP Applications}{tab:erl_apps_behaviours}
\hline
Application & Description & Behaviour Used & Discrepancies\\
\hline
\hline
\multirow{2}{*}{inets}& \multirow{2}{*}{Internet clients and servers}&
gen\_server & 1 \\
\cline{3-4}  
& & tftp & 1 \\
\hline
dist\_ac & distributed application controller & gen\_server & 1 \\
\hline
mnesia & distributed DBMS & gen\_server & 2 \\
\hline
ssh & SSH application & gen\_server & 2 \\
\hline
error\_logger & Stdlib's error logger & gen\_server & 1 \\
\hline
\hline
\multicolumn{3}{|c|}{\textbf{Total discrepancies}} & 8 \\
\hline
\end{fulltable}


All these discrepancies correspond to cases where a callback has a
wider return type than the one described in the relevant
attribute. The most common warning was about the return value of
\genserv's callbaks \texttt{handle\_cast} and \texttt{handle\_info}
which sometimes erroneously included \texttt{\{reply, ...\}}.

Some special attention was given to the \genserv\ module, as its API
returns depended also on the success typings of the
callbacks. Specifically for the API's \texttt{start} and
\texttt{start\_link} functions, \emph{``If callback init/1 fails with
  Reason, the function returns \{error,Reason\}. If callback init/1
  returns \{stop,Reason\} or ignore, the process is terminated and the
  function returns \{error,Reason\} or ignore, respectively.''}. This
was taken into account when calls to these functions were found, as
otherwise false warnings were emitted for the inets and other
applications.

\dr\ emitted some other false warnings as well. These warnings came
from situations where two or more callback functions used the result
of a common underlying function as a reply, or when one callback
called directly another. The example presented in
Listing~\ref{lst:mnesia_false_warning}, from the mnesia application,
falls into the latter category, as \texttt{handle\_info} calls a
specific clause of \texttt{handle\_call}. Using the existing algorithm
for type inference this will result in the inclusion of \emph{all} of
\texttt{handle\_call}'s possible returns in \texttt{handle\_info}'s
return type. This causes the warning shown in the end of the Listing
to be emitted, as \texttt{handle\_call} might also return
\texttt{\{reply,\ldots\}} in other clauses, even though this will
never happen in the call from \texttt{handle\_info} (the returns of
which are in lines 8, 11 and 23 and are all
\texttt{\{noreply,\ldots\}}).

This was the motivation for the second part of this thesis which deals
with cases like this, where a specific call's return type is certainly
narrower than the return type of the function. See
Chapters~\ref{chp:intersection_generate}
and~\ref{chp:intersection_usage}.

\begin{code}{lst:mnesia_false_warning}{A false warning}
handle_call({connect_nodes, Ns}, From, State) ->
    
    ...

    case mnesia_monitor:negotiate_protocol(Check) of
	busy -> 
	    erlang:send_after(2, self(), {connect_nodes,Ns,From}),
	    {noreply, State};
	[] ->
	    gen_server:reply(From, {[], AlreadyConnected}),
	    {noreply, State};
	GoodNodes ->
	    mnesia_lib:add_list(recover_nodes, GoodNodes),
	    cast({announce_all, GoodNodes}),
	    case get_master_nodes(schema) of 
		[] ->
		    Context = starting_partitioned_network,
		    mnesia_monitor:detect_inconcistency(GoodNodes, Context);
		_ -> %% If master_nodes is set ignore old inconsistencies
		    ignore
	    end,
	    gen_server:reply(From, {GoodNodes, AlreadyConnected}),
	    {noreply,State}
    end;

...

handle_info({connect_nodes, Ns, From}, State) ->
    handle_call({connect_nodes,Ns},From,State);

%% Produces the warning:

mnesia_recover.erl:850: The inferred return type of the handle_info/2
callback includes the type {'reply','ok' | {'ok',_},_} which is not a
valid return for the gen_server behaviour

\end{code}

\section{Use of behaviour information to find more race conditions}

\dr\ was recently extended with the ability to detect data races in
Erlang programs~\cite{Races@PADL-10}. This extension makes heavy use
of the dataflow analysis as race conditions appear when a value is
obtained from two separate processes, modified and then written back.

In cases where OTP's behaviours are used, the flow of data is
difficult to monitor because the behaviours' APIs use parameters
obtained in runtime to make calls to functions in the callback
modules. Therefore any values provided as arguments to behaviour API
calls may end up in the callback module and cause a race condition
that is impossible to detect as the call from the API to the callback
is dependent on runtime parameters. A special ``bypass'' mechanism was
added in the race detection that translates calls to the behaviour API
of OTP's behaviours into the respective call to a callback function,
as they are described in the documentation. Examples of such
translations for the \genserv\ module are given in
Table~\ref{tab:gen_server_transl}.

\begin{fulltable}{|c|c|}{OTP's \genserv\ translations}{tab:gen_server_transl}
\hline
Call to API's & ... is translated to a call to callback's\\
\hline
\hline
start\_link, start & init         \\
\hline
call, multi\_call  & handle\_call \\
\hline
cast, abcast       & handle\_cast \\
\hline
\end{fulltable}

This extension caught a specific bug that escaped the existing race
condition analysis. It is currently being tested with other additions
to the race analysis.

\chapter{Intersection Types generation}
\label{chp:intersection_generate}

As described in Section \ref{sct:dialyzer_preliminaries}, \dr\ has two
distinct phases in its analysis: during the first it calculates the
success typings of all the functions and during the second it finds
the discrepancies in their use. In this chapter the calculation of
intersectioned success typings will be presented, leaving their usage
in discrepancy detection for Chapter \ref{chp:intersection_usage}.

\section{Original type system and analysis}

Before describing the design and implementation of the intersection
types as well as the analysis needed to produce and use them, a brief
overview of the existing \er's type system will be given, focusing on
the type of functions and the analysis performed by \dr\ to generate
them. Especially for the analysis, \cite{Elli} explains everything in
far greater detail and is a suggested reading.

\subsection{Type system}

\er's type system includes types for all the basic term sets. Table
\ref{tab:erl_terms} briefly describes these sets and Table
\ref{tab:erl_types} contain the most commonly used \er\ types. More
information on these can be found in \cite{erlman, type_system, Manolis}.

\begin{fulltable}{|c|c|c|}{Categories of Erlang terms}{tab:erl_terms}
\hline
Category & Description & Examples \\
\hline
\hline
Integer & A mathematical integer & $-31$, $0$, $17$, $42$ \\
\hline
Float & A floating point number & $-0.123$, $3.14$ \\
\hline
Atom & A named constant & hello, `World' \\
\hline
Binary & An untyped series of bytes & <<255,0,98>>, <<42>> \\
\hline
Bitstring & An untyped series of bits & <<99,3:2>>, <<1:1,0:1>>, <<4>> \\
\hline
Pid & A handle for talking to an Erlang process & -- \\ % <0.36.0>
\hline
Port & A handle for talking to an external program & -- \\ % \#Port<0.51>
\hline
Reference & A term unique within a runtime environment & -- \\ % \#Ref<0.0.0.62>
\hline
\multirow{2}{*}{Fun} & \multirow{2}{*}{A callable function object}
& fun(X) $\rightarrow$ X + 1 end, \\ && fun lists:reverse/1 \\
\hline
\multirow{2}{*}{Tuple}
    & A compound term with       & \{0,alabama,3.14\}, \\
    & a fixed number of elements & \{answer,42\} \\
\hline
\multirow{2}{*}{List}
    & A compound term with a variable number of   & [1,2,3], [42,answer], \\
    & elements (not necessarily of the same type) & [for,whom,the,bell,tolls] \\
\hline
\end{fulltable}

\begin{fulltable}{|c|c|m{7.7cm}|}{Built-in Erlang types}{tab:erl_types}
\hline
Term Group & Related Types & Represented Terms \\
\hline
\hline
\multirow{5}{*}{Integers}
    & \emph{<Int>} & only a specific integer, \emph{<Int>} (singleton type) \\
    \cline{2-3}
    & \emph{<Lo>}$..$\emph{<Hi>}
    & integers between \emph{<Lo>} and \emph{<Hi>} \\
    \cline{2-3}
    & integer() & all integers \\
    \cline{2-3}
    & non\_neg\_integer() & non-negative integers \\
    \cline{2-3}
    & pos\_integer() & positive integers \\
    \cline{2-3}
    & neg\_integer() & negative integers \\
\hline
Floats & float() & all floats \\
\hline
\multirow{2}{*}{Atoms}
    & \emph{<Atom>} & only a specific atom, \emph{<Atom>} (singleton type) \\
    \cline{2-3}
    & atom() & all atoms \\
\hline
\multirow{3}{*}{Binaries}
    & binary() & all binaries \\
    \cline{2-3}
    & <<>> & only the empty binary (singleton type) \\
    \cline{2-3}
    & <<\_:\emph{<Base>}>> & binaries of length \emph{<Base>} (in bytes) \\
\hline
\multirow{4}{*}{Bitstrings}
    & bitstring() & all bitstrings \\
    \cline{2-3}
    & <<>> & only the empty bitstring (singleton type) \\
    \cline{2-3}
    & <<\_:\_*\emph{<Unit>}>>
    & bitstrings of length $k\times$\emph{<Unit>} (in bits) \\
    \cline{2-3}
    & <<\_:\emph{<B>}, \_:\_*\emph{<U>}>>
    & bitstrings of length \emph{<B>}$\times$\emph{<U>} (in bits) \\
\hline
Pids & pid() & all pids \\
\hline
Ports & port() & all ports \\
\hline
References & reference() & all references \\
\hline
\multirow{4}{*}{Funs}
    & fun() & all functions \\
    \cline{2-3}
    & fun((\ldots) $\rightarrow$ \emph{Type})
    & functions of any arity returning \emph{Type} \\
    \cline{2-3}
    & fun(() $\rightarrow$ \emph{Type})
    & zero-arity functions returning \emph{Type} \\
    \cline{2-3}
    & fun(($T_1$,\ldots,$T_N$) $\rightarrow$ \emph{R})
    & N-arity functions accepting arguments of types $T_1$,\ldots,$T_N$ and
      returning \emph{R} \\
\hline
\multirow{3}{*}{Tuples}
    & tuple() & all tuples \\
    \cline{2-3}
    & \{\} & only the zero-size tuple (singleton type) \\
    \cline{2-3}
    & \{$Type_1$,\ldots,$Type_N$\}
    & tuples of N elements, of types $Type_1$,\ldots,$Type_N$ \\
\hline
\multirow{3}{*}{Lists}
    & [] & only the empty list (singleton type) \\
    \cline{2-3}
    & [\emph{Type}] & lists with elements of type \emph{Type} \\
    \cline{2-3}
    & [\emph{Type},\ldots]
    & non-empty lists with elements of type \emph{Type} \\
\hline
\multirow{3}{*}{---}
    & any() & all Erlang terms \\
    \cline{2-3}
    & none() & no terms (special type) \\
    \cline{2-3}
    & $T_1$ | $T_2$ | \ldots\ | $T_N$
    & the union of all terms represented by $T_1$, $T_2$, \ldots, or $T_N$ \\
\hline
\end{fulltable}

\subsubsection{Function type}
\label{sct:orig_fun_type}

Some special attention should be given to the form of the function
type. As shown in Table \ref{tab:erl_types}, a function's type
consists of two parts:
\begin{enumerate}
\item The first part describes the type of the function
  arguments. This can be either a product of specific length, with one
  type for each of the function's arguments, or the type \emph{any} if
  we have no information about the number of arguments.
\item The second part describes the return type of the function. This
  can be a regular type or the special type \emph{unit} for functions
  that are not supposed return\footnote{This is the case for example
    in a function that implements a server's main loop}.
\end{enumerate}

\subsubsection{Type operations}
\label{sct:orig_type_operations}

Working with types requires special operators. The most important of
them will be presented now (as these were the ones which were mainly
affected by the introduction of intersection types), along with the
particular usage of them on function types.
\begin{description}
\item[Supremum:] The supremum of two types is the smallest type that
  contains them both. \er\ supports union types as shown in Table
  \ref{tab:erl_types} as well as special unions for common types (like
  atoms, integers and tuples). \dr\ may overapproximate a union type
  in cases where detail exceeds a certain level to keep the success
  typings analysis efficient. Such is the case with large sets of
  atoms for example: the type of the single-character atoms
  corresponding to the lowercase letters of the English alphabet is
  \atom\ and not \emph{a | b | ... | z}.

  The original calculation of two function types' supremum is the main
  reason for our inability to detect errors such as the one presented
  in Section \ref{sct:intersection_preliminaries}. As the type system
  has one field for the success typings of the arguments and one for
  the function's return, supremum simply creates the union of the
  arguments and the result and stores them in the respective
  fields. This produces wider success typings than desired (in the
  example shown in Listing \ref{lst:fun_sup_example} a function that
  takes an \atom\ and returns a \emph{number} is included in the type,
  even though none of the original members includes it). If the
  functions have different arity the domain type collapses into
  \emph{any}.

\begin{console}{lst:fun_sup_example}{Original function supremum example}
Supremum:
Type A : fun((atom())   -> atom())
Type B : fun((number()) -> number())
Result : fun((atom() | number()) -> atom() | number())
\end{console}

\item[Infimum:] The infimum of two types is the biggest type that is
  contained in both.

  In function types of the same arity this means that the infimum will
  have the infimum type for each argument and the infimum of the
  return types.

\item[Equality:] As there are no aliases two types are equal if and
  only if they are syntactically equal. This applies even to the
  internal representation which keeps every particular set (like
  \atoms\ or \emph{integers}), as well as mixed unions, ordered.

  Functions types are no exception and should also be syntactically
  equal to be equal.

\item[Is Subtype:] This operation is broken down to the calculation of
  the infimum of the two types and the check for its equality with the
  subtype candidate.
\end{description}

There are also special operations for function types:

\begin{description}
\item[Function domain:] Returns a list of types, one for each of the
  function's arguments.
\item[Function range:] Returns the range of the function.
\end{description}

\subsection{Original success typing analysis}
\label{sct:orig_analysis}

\dr\ begins the analysis by finding the calls between functions and
creating the respective callgraph. From this callgraph a partial
ordering of the functions is obtained and all functions' success
typings are calculated beginning from those who have no calls and
building on top of them.

The success typings analysis assigns a type variable to each of the
code's original variables and stores a mapping from each variable to a
type. Functions get two variables, one primary and a second one to be
used in self-calls and SCC calls. After that the code is traversed to
generate constraints and subsequently these are processed to obtain
the final type.

A brief presentation of the main types of constraints and the
processing algorithm will be included here to make illustrating the
differences introduced in the new analysis easier. For further details
you can refer to the appropriate sections in the bibliograhy
\cite{Elli, SuccessTypings@PPDP-06}.

\subsubsection{Constraints}
\label{sct:orig_constraints}

The constraints belong to one of the following kinds:
\begin{description}
\item[Simple constraints:] The simplest form of constraint states that
  a certain type should be equal to another or subtype of
  another. This is a natural requirement for function arguments for
  example, which must be subtypes of the corresponding success typing,
  calculated earlier in the analysis.
\item[Conjunctive lists:] The constraints generated from subsequent
  statements are stored in conjunctive lists as all must be satisfied
  at the same time. In simple functions the final constraint might be
  a conjunctive constraint list with simple constraints as elements.
\item[Disjunctive lists:] When branches of any kind are present in the
  code, each side of the branch generates a conjunctive list and all
  these lists are combined under a disjunction. After processing each
  of the conjunctions, the types for the disjunction are calculated by
  getting the supremum of the types for each variable on each branch.
\item[Constraint references:] Funs without name generate these. These
  are special and not really relevant with this extension so they are
  simply mentioned for completeness.
\end{description}

\begin{console}{lst:constraints}{Constraint examples}
%%Sample code

bar(1) -> 5;
bar(2) -> 10.

foo(a) -> b;
foo(X) ->
  Y = bar(X),
  Y*X.

%% Supposing we have alredy found the success typing:
%% bar(1 | 2) -> 5 | 10
%% The constraints for foo are:

Conjunctive List 1:                <- All the constraints for foo
 * var(1) eq fun(var(2)) -> var(3) <- Tying foo type to it's args and ret
 * Disjunctive List 2:             <- Due to the two clauses
 *  * Conjunctive List 3:          <- Constraints for the first clause
 *  *  * var(2) eq a
 *  *  * var(3) eq b
 *  * Conjunctive List 4:          <- Constraints for the second clause
 *  *  * var(2) sub 1|2            <- X (var(2)) is used as argument of bar
 *  *  * var(4) sub 5|10           <- A hidden variable (var(4)) for the result
 *  *  * var(5) eq var(4)          <- Assign the result to Y (var(5))
 ...
\end{console}

In Listing \ref{lst:constraints} a real example is given. As you can
see, the constraints of each function are collected in a main
conjunctive constraint list. In this list there exist some notable
constraints:

\begin{enumerate}
  \item \textbf{Generic function constraint:} This constraint has the
    form of the first element in the conjunctive list 1 of Listing
    \ref{lst:constraints}. It's purpose is to bind the function's type
    variable to the ones of the arguments and the result. In the
    example \emph{var(1)} is the type variable of a function with one
    argument (with type variable \emph{var(2)}) whose return type is
    \emph{var(3)}. This constraint is the actual constructor of the
    function's type.
  \item \textbf{Refined function constraint:} This constraint comes
    from the dataflow analysis and restricts the whole type of
    unexported functions according to the actual calls that are
    present within the module. For more information on function
    refinement see \ref{sct:dialyzer_preliminaries}. This constraint
    is omitted when the function is exported or dataflow has not yet
    been performed.
  \item \textbf{Constraints from clauses:} If the function has
    clauses, the third constraint in the list is a disjunctive list of
    the constraints generated in each of them (an example is the
    disjunctive list 2 in \ref{lst:constraints}). If the function has
    only one clause the constraints of it are added in the main
    conjunctive list as is.
\end{enumerate}

The previous constraints are present in the form described above in
every main conjunctive list. Some other forms of constraints that are
present in almost every function are these:

\begin{enumerate}
  \item \textbf{Branches:} Branches such as \emph{case} statements
    generate disjunctive lists, just like clauses do.
  \item \textbf{Function calls:} These produce a conjunctive list,
    requiring both the result's and the actual parameters' type
    variables to be subtypes of the respective success types.
  \item \textbf{Self and SCC calls:} These are treated specially:
    Initially these calls are supposed to fail. On subsequent
    iterations the types calculated in the previous step are used to
    extend the types of the argument and the result. In this way we
    begin to extract the type from clauses that are sure to return and
    build on top of them to find wider success typings. The
    overapproximations mentioned in \ref{sct:orig_type_operations}
    (Supremum operator) make this procedure efficient, as after a
    certain limit the types collapse into generic ones.
\end{enumerate}

\subsubsection{Processing}
\label{sct:orig_processing}

Processing is a fixpoint procedure when it comes to anything but
simple constraints. The latter simply restrain any variables they
contain according to the operation they contain (equality or subtype)
and store the result in the mapping. Lists and refs store the old
mappings and compare the new ones against them to find a fixpoint as
each element may affect others. Self-recursive functions and SCCs also
have special treatment as the previously calculated types are fed back
in to be further processed in the self or scc-related calls.

\section{Intersection types}
\label{sct:intersection_types}

In order to generate and use intersection types changes were required
in both the type system and \dr's analysis.

\subsection{Changes in the type system}
\label{sct:intersection_type_system}

\subsubsection{Representation}
\label{sct:intersection_representation}

We need the ability to store multiple domains with the respective
ranges. Therefore we will substitute the original two fields in the
function type with a list of tuples of arity 2. Each tuple will
contain a domain and a range and will be referred as a \emph{clause}.
\dr\ took advantage of the simple old form and stored the type of the
function in the plt using a tuple containing the two old parts. This
had to be changed and a proper full function type to be stored
instead.

\subsubsection{Semantics}
\label{sct:intersection_semantics}

For simplicity the order of the clauses will have no special
meaning. This is the main difference with the ordinary \er's function
clauses, where pattern matching is used to select one and execute it
while the others that follow are ignored. The consequence is that for
every operation described all the clauses have to be taken into
account.

This also changes the semantics of the final type. The syntax is
similar to that of specs, \textbf{BUT} while in specs overlapping is
not permitted (see Section \ref{sct:dialyzer_preliminaries}), here it
is allowed and the return type of each specific call is calculated by
taking the supremum of the return types of \textbf{EVERY} clause whose
domain overlaps \footnote{This means that the infimum of every
  argument's type in the call and the respective type in the clause is
  not \none} with the inferred types of the arguments in the call.

\subsubsection{Operations}

The operations described in \ref{sct:orig_type_operations} are
modified as follows, with regard to function types:

\begin{description}
\item[Supremum:] If the functions have different arities we maintain
  the old behaviour, collapsing the domains to \any\ and the range to
  the supremum of ranges. If the functions have the same arity, we
  simply add their clauses together to form the supremum. This
  produces an exact supremum. No issue arises in cases of duplicate
  domains as all the clauses are taken into account for further
  calculations (in fact clauses with equal domains are combined into
  one, as described later in this section).
\item[Infimum:] As each function might have more than one clauses,
  infimum is performed per clause. This means that each clause is
  compared against all the clauses of the other function and those who
  have infima that do not have \none\ as an argument or return are
  kept in the result.
\item[Reduction of clauses:] The calculation of supremum and infimum
  is almost certain to produce types that are verbose. This makes the
  clauses list big, requirng both memory to store it and time to
  perform further calculations. An extra step was therefore introduced
  to reduce the number of clauses. Three separate methods of reduction
  are used:
  \begin{enumerate}
    \item \textbf{Combine same domains:} Clauses with the same domain
      should combine their ranges with supremum. An example of this
      technique is given in Example 1 in Listing
      \ref{lst:clauses_reduction_example}.
    \item \textbf{Combine ranges:} Clauses with the same range may
      also be combined if this does not overapproximate the
      domains. This happens when the domains differ in exactly one
      position and the supremum of the types that differ contains only
      the original types (we don't have overapproximations). This
      computation requires a fixpoint termination condition as further
      reduction may be possible in a successive pass. Examples 2 and 3
      in the same Listing illustrate this technique.
    \item \textbf{Remove subclauses:} If both the domain and the range
      of a clause are subtypes of another clause's respective domain
      and range we can remove the clause, as anything using it will
      also use the superclause. This causes intersections to lose
      power in cases where catch-all clauses with return type
      \any\ are present in the code, as these will cause all the rest
      to be absorbed in them. A solution to this issue is proposed in section
      \ref{sct:negative_types}.
      \begin{console}{lst:clauses_reduction_example}{Clauses reduction examples}
        Example 1: Same domains in supremum
        -----------------------------------
        Type A : fun((a) -> b)
        Type B : fun((a) -> c)
        Result : fun((a) -> b; (a) -> c)
        Reduced: fun((a) -> b | c)
        -----------------------------------
        Example 2: Same ranges in supremum
        -----------------------------------
        Type A : fun((a) -> c)
        Type B : fun((b) -> c)
        Result : fun((a) -> c; (b) -> c)
        Reduced: fun((a | b) -> c)
        -----------------------------------
        Example 3: Same ranges, 2 passes
        -----------------------------------
        Initial  : fun((a,c) -> e; (a,d) -> e; (b,c) -> e; (b,d) -> e)
        1st pass : fun((a,c|d) -> e; (b,c|d) -> e)
        2nd pass : fun((a|b,c|d) -> e)

      \end{console}
  \end{enumerate}
\item[Sorting of clauses:] The clauses are sorted according to the
  default ordering of \er's terms both before and after the reductions
  are performed to control both the order of the reductions and the
  final result. This destroys any relation between original clauses in
  the code and resulting success typings but it's important as it
  normalizes the type and maintains the desired property of ``equality
  implies syntactic equality''.
\item[Equality and subtyping:] Equality maintains it's simple, syntax
  based check. The reasons behind this will become clearer after the
  presentation of the changes in the inference algorithm. Clause
  sorting is essential to maintain this property. Subtyping is also
  calculated as described in Section \ref{sct:orig_type_operations}.
\item[Function range:] As described in Section
  \ref{sct:intersection_semantics}, when asking for a function's range
  we may provide information about the argument types and retrieve a
  narrower type.
\end{description}

Other special type functions had to be modified as well to be
compatible with the new representation. What is important to mention
is that in any case where the inner types might be modified (as in
substitutions of opaque types by simple ones and limitation of type
depth), reduction had to be performed as well to ensure the syntactic
equality.

\subsection{Analysis}
\label{sct:intersection_analysis}

The initial steps of the analysis are not changed. Functions are
sorted as described in \ref{sct:orig_analysis} and the success typings
are calculated per SCC. Changes are introduced in both the generation
of constraints and the processing of them to produce the success
typings. As the most important change is implemented in the processing
we will reverse the order of the presentation.

\subsubsection{Changes in constraint processing}
\label{sct:intersect_constr_process}

As we already described in Section \ref{sct:orig_constraints}, each
function has all the constraints organized in a conjunctive list. This
list contains disjunctive lists whenever a branch is present in the
code. Moreover each of these has conjunctive lists with local maps
which are checked for fixpoint and remain unaffected from the other
sides of the branch. Therefore, the simplest and most natural way to
maintain the relation between the various type variables (including
those belonging to the arguments and the result) in each branch is
within the local maps themselves.

Although the values of the type variables are being kept separate in
that way, the type of the function itself is constructed in the main
conjunctive list taking into account the supremum of all the values,
as this is the correct way to handle the types in a disjunctive
list. We need to ``push'' the constraint that binds the type variables
of the arguments and the result with the type variable of the function
into each local map. Taking the \emph{disjunctive normal form} of the
original constraint list accomplishes this goal in a natural way and
separates all the interleavings where nested disjunctive lists are
present (for example a case statement in a branch of a multi-clause
function). An example is provided in Listing \ref{lst:normal_form}. In
this way we generate the correct partial type in every branch and
combine them all in the end using supremum which maintains the
separation. The disjunctive normal form is already used in the
generation of constraints from \emph{guards} to gain precision.
Another benefit we gain is that whenever the function is too
complicated, the calculation of the disjunctive normal form can detect
it and return the original constraint list which when further
processed will return the old-fashioned collapsed type.

\begin{console}{lst:normal_form}{Normal form example}
Conjunctive List 1:
 * var(1) eq fun(var(2)) -> var(3)
 * Disjunctive List 2:
 *  * Conjunctive List 3:
 *  *  * var(2) eq a
 *  *  * var(3) eq b
 *  * Conjunctive List 4:
 *  *  * var(2) sub 1|2
 *  *  * var(4) sub 5|10
 *  *  * var(5) eq var(4)

Disjunctive normal form of the list:

Disjunctive List 1:
 * Conjunctive List 2:
 *  *  var(1) eq fun(var(2)) -> var(3)
 *  *  var(2) eq a
 *  *  var(3) eq b
 *  Conjunctive List 3:
 *  *  var(1) eq fun(var(2)) -> var(3)
 *  *  var(2) sub 1|2
 *  *  var(4) sub 5|10
 *  *  var(5) eq var(4)
\end{console}

\subsubsection{Changes in constraint generation}
\label{sct:intersect_constr_generation}

To gain the benefits of the disjunctive normal form when function
calls are present we need to generate a disjunctive list as a
constraint when processing them. The way this should be done is
obvious: for every clause in the function type a separate conjunctive
list is to be generated, binding the argument and result type
variables to the respective success types in the clause. These
conjunctive lists are then placed in a disjunctive list and the
constraint is ready to be handled by the normalization (see Listing
\ref{lst:call_disjunction} for an example).

\begin{console}{lst:call_disjunction}{Disjunction for function calls}
%%Sample code

bar(1) -> 5;
bar(2) -> 10.

foo(a) -> b;
foo(X) ->
  Y = bar(X),
  Y*X.

%% Supposing we have alredy found the success typing:
%% bar(1) -> 5; (2) -> 10
%% The new constraints for foo are:

Conjunctive List 1:
 * var(1) eq fun(var(2)) -> var(3)
 * Disjunctive List 2:
 *  * Conjunctive List 3:
 *  *  * var(2) eq a
 *  *  * var(3) eq b
 *  * Conjunctive List 4:
 *  *  * Disjunctive List 5:
 *  *  *  *  Conjunctive List 6:
 *  *  *  *  *  var(2) sub 1
 *  *  *  *  *  var(4) sub 5
 *  *  *  *  Conjunctive List 7:
 *  *  *  *  *  var(2) sub 2
 *  *  *  *  *  var(4) sub 10
 *  *  * var(5) eq var(4)
 ...
\end{console}

This is simple in cases where the success typing of the called
function is calculated and fixed, as the generation of the disjunctive
constraint can take place immediately. The hard case is the
self-recursive functions along with those that belong in SCCs. For
these we introduced a new \emph{dynamic constraint} which is to be
substituted before the calculation of the disjunctive normal form by
the disjunctive list derived by the latest success type of the
respective function.

The usage of intersections simplified the generation of constraints
from \emph{contracts} as well, as a similar disjunctive list can be
used for them as well.

\subsubsection{Changes in refinement}
\label{sct:intersect_refinement}

The use of intersectioned types allowed for a small improvement in the
refinement of success typings as well. The previous approach was
\emph{monovariant} in the sense that all the calls to the unexported
functions were found and the types of the actual arguments were
combined in a union that restricted the success typing. This
restriction was taken into account in a new calculation of the success
typing by solving the default constraints with the addition of the
refinement constraint.

Allowing for intersection types, we can keep each call separate and
expect better results from the refinement. This is closer to a
\emph{polyvariant} control flow analysis.

As an example take the simple reversal of lists, presented in Listing
\ref{lst:reverse}. The one-argument function is exported while the
other is kept local and can therefore be refined. The initial success
typing is very generic because the first clause doesn't restrict the
second argument. Using the union of the types from the two calls we
learn that the second argument is a list, so the result must be a list
as well but no distinction is made. Only by separating the calls and
solving each case separately can we obtain the maximum information
from this code.

\begin{console}{lst:reverse}{Refinement of the success typing of reverse}
reverse(List) ->
  reverse(List, []).

reverse(    [], Acc) -> Acc;
reverse([H| T], Acc) -> reverse(T, [H| Acc]).

%% Initial success typing for reverse/2 is:
-spec reverse([_], _) -> any().

%% Using (_,[_]) as a refinement for the arguments yields:
-spec reverse([_], [_]) -> [_].

-spec reverse([_]) -> [_].

%% Using separate (_,[]) and (_,[_,...]) as a refinement 
%% for the arguments yields:
-spec reverse([_,...],[_]    ) -> [_,...];
             ([]     ,[_,...]) -> [_,...]; 
             ([]     ,[]     ) -> [].

-spec reverse([_,...]) -> [_,...];
             ([]     ) -> [].
\end{console}

\chapter{Using Intersection Types}
\label{chp:intersection_usage}

\section{Testing with PropEr}

Testing the implementation of intersection types proved to be an
excellent opportunity for another tool developed in SoftLab to show
its shine: PropEr \cite{Manolis}. Types are inherently an abstract
data type with its own operators and properties that should be
satisfied by them. A brief overview of the testing using PropEr will
be given in this section.

\subsection{Generating random function types}

Property-based testing requires a generator for random input for the
tests. Using PropEr we were able to create a generator for
intersectioned function types with ease taking into consideration
parameters as:
\begin{itemize}
\item Covering all the simple types for both arguments and result
\item Testing operators for both same arity and differing arity functions
\item Helping PropEr's reduction with simple primitive types (like 'a'
  for atoms)
\item Pretty printing of failing tests using PropEr's ?WHENFAIL directive
\end{itemize}

\subsection{Properties of function types}

As the actual implementation was a result of experimentation, trivial
properties as well as stronger ones were tested. Some of them:

\begin{enumerate}
\item Simple function types combine correctly into an intersection
\item The supremum/infimum of F with F is equal to F
\item Subtracting F from F yields \none
\item F is subtype of F
\item If Inf is the infimum of F1 and F2 then Inf is a subtype of both of
  them
\item If Sup is the supremum of F1 and F2 then F1 and F2 are subtypes of
  Sup
\item If Sup is the supremum of F1 and F2, Inf1 is the infimum of Sup
  and F1, Inf2 is the infimum of Sup and F2 then Inf1 is equal to F1
  and Inf2 is equal to F2
\item The supremum/infimum of F1 and F2 is equal to the
  supremum/infimum of F2 and F1
\item If Inf is the infimum of F1 and F2 then the infimum of F1 and
  Inf is equal to Inf
\item If F1 is subtype of F2 and F2 is subtype of F1 then they are equal
\end{enumerate}

\subsection{Side results}

Using these properties and some early implementations with no
syntactic equality (as presented in Section
\ref{sct:orig_type_operations}) we found a two minor omission in the
type system of Erlang, in an operator that returned all the simple types
contained in a composite type:
\begin{enumerate}
\item Arbitrary lists didn't return as simple types the empty list and
  the nonempty list with the same contents
\item Unspecified numbers didn't break down to unspecified integers
  and unspecified floats.
\end{enumerate}

Fixing the first omission led to the detection of many loose contracts
in OTP (where nonempty list were sure to be returned whereas the
contract included the empty list as a possible return).

\subsection{Conclusion}

Using PropEr was a very creative and fun experience as ideas could be
tested quickly against the properties required.

\section{Performance issues}

Sparsely in the previous secrions we mentioned the need of limits in
the geneneration of intersections. These limits were imposed when the
analysis was under risk to become needlessly time consuming. The
recent parallelization of Dialyzer \cite{Ypatia} should push these
limits further.
\begin{description}
\item[Number of clauses:] In the calculation of the disjunctive normal
  form, in cases where too many clauses or deep nesting of branches is
  present it is possible for the normal form to have too many
  branches. A limit was put to the number of them to maintain both
  efficiency and usability, as a very long success typing would alse
  be impractical to present to the user.
\item[Size of SCCs:] The analysis of SCCs requires a separate fixpoint
  and the substitution of the \emph{dynamic constraints} mentioned in
  Section \ref{sct:intersect_constr_generation} each time using the
  latest type. This becomes impractical when the SCC is particularly
  big so another limit was placed in the size of it.
\item[Iterations in SCCs and self-recursive:] Before intersection
  types \dr's overapproximations guaranteed that fixpoint would be
  reached in a reasonable amount of iterations. This is no longer the
  case. Consider the example in Listing \ref{lst:self_rec_id_fun}. The
  iterative process will infer that the function returns 0 for input
  0, 1 for input 1 and so forth, without any reason to stop or any
  mean to find a fixpoint (previously after a few iterations both
  success types would collapse into \emph{integer}).
  \begin{console}{lst:self_rec_id_fun}{A self recursive numeric identity function}
    id(0) -> 0; id(N) -> 1 + id(0)
  \end{console}
\end{description}

In all these cases we simply skip the tranformation to the normal form
after a fixed number of iterations. This causes the success typing to
collapse as the combination of the arguments' and the return types
happens in the end (as described in Sections \ref{sct:orig_analysis}
and \ref{sct:intersect_constr_process}).

\section{Intersection analysis results}

The actual usage of the success typings for discrepancy detection
happens in a final dataflow pass on the code under inspection. We
won't go into detail here on the various warnings that may be emitted
as these are covered in detail in the relevant publications
\cite{Elli, SuccessTypings@PPDP-06,
  springerlink:10.1007/978-3-540-30477-7_7}.

The only change we implemented there is the substitution of the
generic lookup for the return type of function calls with a lookup
that takes into consideration the types of the arguments. In this way
we can easily detect discrepancies like the one in the initial example
(\ref{sct:intersection_preliminaries}).

\subsection{Generic discrepancies}

Using the extended \dr\ the results presented in Table
\ref{tab:intersection_results} were found. They are divided in the
following categories:

\begin{description}
\item[Failing calls:] These are calls that are certain to fail. This
  is usually the result of a particular combination of arguments. This
  category includes calls that are supposed to fail but no spec is
  provided so that dialyzer knows not to worry (See Listing
  \ref{lst:failing_call}.
\item[Unneeded cases:] These are \emph{case} statements that have an
  unneeded error-catching or catch-all clause (See Listing
  \ref{lst:unneeded_case}.
\item[Exit calls:] Calls that result in an \emph{erlang:exit}. These
  come from error-handling functions that do not always fail. When
  such calls are present, user should specify that the function may
  not return (See Listing \ref{lst:exit_call}). 
\item[Nonmatching clauses:] These are nonmatching clauses that are not
  catch-alls.
\item[Deriving warnings:] In some cases a root failure may cause
  several more warnings to be emitted. These are listed with this
  category. An example is functions that won't be called due to an
  error earlier in the flow of control. Fixing the root cause will
  eliminate these warnings as well.
\end{description}

\begin{fulltable}{|c|c|c|c|}{New Discrepancies in OTP Applications}{tab:intersection_results}
\hline
Application & Description & Category & Discrepancies\\
\hline
\hline
asn1 & Abstract Syntax Notation 1 tools & Exit call & 1 \\
\hline
auth & Network Authentication Server & Deriving warnings & 9 \\
\hline
edoc & Documentation generator & Failing call & 1 \\
\hline
\multirow{2}{*}{erts} & \multirow{2}{*}{Erlang Run-Time System} &
Failing calls & 2 \\
\cline{3-4}
& & Deriving warnings & 17 \\
\hline
file & File Interface Module & Deriving warnings & 25 \\
\hline
hipe & High Performance compiler & Unneeded case & 1 \\
\hline
\multirow{3}{*}{inets}& \multirow{3}{*}{Internet clients and servers} &
Failing calls & 2 \\
\cline{3-4}
& & Exit call & 1 \\
\cline{3-4}
& & Nonmatching clause & 1 \\
\hline
\multirow{2}{*}{mnesia} & \multirow{2}{*}{distributed DBMS}
 & Exit calls & 7 \\
\cline{3-4}
& & Unneeded case & 1 \\
\hline
ssh & SSH application & Unneeded case & 1 \\
\hline
\multirow{3}{*}{ssl} & \multirow{3}{*}{Interface for Secure Socket Layer} &
Failing calls & 2 \\
\cline{3-4}
 & & Unneeded case & 1 \\
\cline{3-4}
& & Deriving warnings & 1 \\
\hline
\multirow{2}{*}{tv} & graphical examination of & \multirow{2}{*}{Unneeded case} & \multirow{2}{*}{1} \\
& ETS and Mnesia tables & & \\
\hline
\hline
\multicolumn{3}{|c|}{\textbf{Total original errors}} & 22 \\
\hline
\end{fulltable}


\begin{console}{lst:failing_call}{A call that will surely fail}
httpd_request_handler.erl:439: The call
httpd_response:send_status(ModData::#mod{data::[], method::[any()],
  request_line::nonempty_maybe_improper_list(),
  parsed_header::[any()], connection::boolean()},501, [1..255,...])
will never return since it differs in the 2nd and/or 3rd argument from
the success typing arguments: (#mod{socket_type::'ip_comm' |
  {'essl',_} | {'ossl',_} | {'ssl',_}},100 | 304 | 400 | 408 | 413 |
416 | 500 | 503,any()) or (#mod{socket_type::'ip_comm' | {'essl',_} |
  {'ossl',_} | {'ssl',_}},301 | 403 | 404 | 414,[any()]) or
(#mod{socket_type::'ip_comm' | {'essl',_} | {'ossl',_} |
  {'ssl',_}},400 | 401 | 412,'none') or (#mod{socket_type::'ip_comm' |
  {'essl',_} | {'ossl',_} | {'ssl',_}},501,{atom() |
  [any()],[any()],[any()]})

%% A rather esoteric warning about the 501 Not Implemented HTTP status message
\end{console}

\begin{console}{lst:unneeded_case}{A redundant catch-all clause}
con_desc(E) ->
    case cerl:type(E) of
	cons -> {?cons_id, 2};
	tuple -> {?tuple_id, cerl:tuple_arity(E)};
	binary -> {?binary_id, cerl:binary_segments(E)};
	literal ->
	    case cerl:concrete(E) of
		[_|_] -> {?cons_id, 2};
		T when is_tuple(T) -> {?tuple_id, tuple_size(T)};
		V -> {?literal_id(V), 0}
	    end;
	_ ->
	    throw({bad_constructor, E})
    end.

%% Produces the warning

cerl_pmatch.erl:338: The variable _ can never match since previous clauses
completely covered the type 'binary' | 'cons' | 'literal' | 'tuple'
\end{console}

\begin{console}{lst:exit_call}{A path that ends in a call to \emph{exit}}
...
check_if_valid_tag(<<>>, _, OptOrMand) ->
    check_if_valid_tag2(false,[],[],OptOrMand);
...

check_if_valid_tag2(_Class_TagNo, [], Tag, MandOrOpt) ->
    check_if_valid_tag2_error(Tag,MandOrOpt);

...

check_if_valid_tag2_error(Tag,mandatory) ->
    exit({error,{asn1,{invalid_tag,Tag}}});
check_if_valid_tag2_error(Tag,_) ->
    exit({error,{asn1,{no_optional_tag,Tag}}}).

%% Produces the warning

asn1rt_ber_bin.erl:2200: The call
asn1rt_ber_bin:check_if_valid_tag2('false',[],[],OptOrMand::any()) will never
return since it differs in the 2nd argument from the success typing arguments:
('false' | {'APPLICATION',_} | {'CONTEXT',_} | {'PRIVATE',_} |
{'UNIVERSAL',_},nonempty_maybe_improper_list(),[] | {_,_,_},any())
\end{console}

\subsection{Behaviour related results}

The introduction of intersection types removed all the false positives
from the previously collected results on behaviour usage. The reason
for this is precisely the one described in Section
\ref{sct:behaviour_discrepancies}: intersectioned types were assigned
to the functions that handled all the requests and the analysis was
able to discern whether a particular call could end in each result
instead of assuming they were all possible. None of the other warnings
were affected.

\subsection{Bonus results}

In Section REF we mentioned how the correction of a small omission
produced warnings about overspecified functions. Another such small
error regarded the relation between the \none\ and \emph{unit}
types. The infimum of the two was considered to be \emph{unit}. The
correction of this error unearthed a heap of \emph{``Function X has no
  local return''}.

\chapter{Related and Further work}

\section{Related work}
\label{sct:related_work}

There is almost no other known cases where tools need to make checks
like the ones implemented for \er's \emph{behaviours}, either in their
generic usage or in the detection of races through
them. Object-oriented languages that use the equivalent \emph{abstract
  classes}, \emph{virtual methods} and \emph{interfaces} have static
typing (C++, Java, OCaml, ...) which ensures the fitting of
implementations in every case.

On the subject of intersection types in a dynamically typed
programming language, only DRuby is a known analog and is presented
below. Other related work has been centered around Dylan and
JavaScript and is briefly mentioned as well.

The changes in the refinement procedure presented in Section
\ref{sct:intersect_refinement} are related with many formal approaches
to control flow analysis

\subsection{Diamondback Ruby (DRuby)}

Diamondback Ruby (DRuby)\cite{druby} is a recent tool that blends
Ruby's dynamic type system with a static typing discipline. It uses a
similar approach as \dr\ generating constraints for the variables then
applying a set of rewrite rules exhaustively. Intersection types are
included from the beginning in it's type system, but it cannot infer
them automatically. As a result they need to be annotated by the
developer placing them on the same level as \dr's contracts
(\emph{specs}).

\subsection{Dylan and JavaScript}

Dylan is a dynamically typed object-centered programming language
inspired by Common Lisp and ALGOL. In a recent publication Mehnert
proposed an extension providing function types and parametric
polymorphism to the language\cite{dylan}. The function types
specialize from the previous generic ones but do not include
intersections. Powerful paramteric polymorphism is provided though.

JavaScript is the main scripting language for Web browsers, and it is
essential to modern Web applications. Applying type analysis to
JavaScript is a subtle business because, like most other scripting
languages, JavaScript has a weak, dynamic typing discipline which
resolves many representation mismatches by silent type conversions. In
their publication\cite{javascript} Jensen, M\o ller and Thiemann
develop such a type analyzer which like \dr\ is fully automatic
\emph{but} is designed for \emph{soundness} with regard to the absense
of certain errors.

\dr's success typings approach is mentioned in both these attempts to
provide static typing and discrepancy detection to dynamically typed
languages.

\section{Further work}
\label{sct:further_work}

\subsection{Behaviours}
\subsubsection{Automatic bypass of API for race detection}

The ``bypass'' mechanism was designed to be extensible, allowing other
behaviour API's to be connected with the respective callbacks. Anyone
with a better understanding of the other behaviours may document the
rest OTP's behaviours easily by adding code in
\emph{dialyzer\_behaviours} module. It would be even better if this
connection was tranferred in each behaviour's file, as a new attribute
or as an extension on the \emph{callback} attribute that was
introduced in this thesis.

\subsection{Intersections}

\subsubsection{Negative Types}
\label{sct:negative_types}

The next logical extension to the type system would be negative
types. Examples would be \emph{``any term except the integer 42''} or
\emph{``any atom except a''}. With this infrastructure, when
\dr\ generates disjunctive lists it will be able to eliminate the
types already covered in previous clauses in the following ones. Thus
a ``catch-all'' will not have type \any\ but \emph{``anything but X, Y
  and Z''} where \emph{X, Y} and \emph{Z} will be the types already
covered by previous clauses.

\subsubsection{Tighter coupling between type and code}

Using normal form and sorting on the function's type has an impact on
the relation between the type of the function and the actual code that
generated it. Even though it was easy to find the cause of all the
warnings emitted when intersection types were used, it might be better
to narrow down a warning to a particular clause instead of the generic
pointer to the first line of the function.


% \backmatter
\cleardoublepage % start at the next odd page
\phantomsection % correct hyperlinking
\addcontentsline{toc}{chapter}{\bibname} % add bibliography section to toc
\bibliography{references}
\bibliographystyle{plain} % plain/abbrv/alpha/abstract/apalike/...
% \include{glossary}
% \chapter{Appendix}
% \printindex

\end{document}
