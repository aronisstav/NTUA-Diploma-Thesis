\chapter{Using Intersection Types}
\label{chp:intersection_usage}

\section{Testing with PropEr}

Testing the implementation of intersection types proved to be an
excellent opportunity for another tool developed in SoftLab to show
its shine: PropEr \cite{Manolis}. Types are inherently an abstract
data type with its own operators and properties that should be
satisfied by them. A brief overview of the testing using PropEr will
be given in this section.

\subsection{Generating random function types}

Property-based testing requires a generator for random input for the
tests. Using PropEr we were able to create a generator for
intersectioned function types with ease taking into consideration
parameters as:
\begin{itemize}
\item Covering all the simple types for both arguments and result
\item Testing operators for both same arity and differing arity functions
\item Helping PropEr's reduction with simple primitive types (like 'a'
  for atoms)
\item Pretty printing of failing tests using PropEr's ?WHENFAIL directive
\end{itemize}

\subsection{Properties of function types}

As the actual implementation was a result of experimentation, trivial
properties as well as stronger ones were tested. Some of them:

\begin{enumerate}
\item Simple function types combine correctly into an intersection
\item The supremum/infimum of F with F is equal to F
\item Subtracting F from F yields \none
\item F is subtype of F
\item If Inf is the infimum of F1 and F2 then Inf is a subtype of both of
  them
\item If Sup is the supremum of F1 and F2 then F1 and F2 are subtypes of
  Sup
\item If Sup is the supremum of F1 and F2, Inf1 is the infimum of Sup
  and F1, Inf2 is the infimum of Sup and F2 then Inf1 is equal to F1
  and Inf2 is equal to F2
\item The supremum/infimum of F1 and F2 is equal to the
  supremum/infimum of F2 and F1
\item If Inf is the infimum of F1 and F2 then the infimum of F1 and
  Inf is equal to Inf
\item If F1 is subtype of F2 and F2 is subtype of F1 then they are equal
\end{enumerate}

\subsection{Side results}

Using these properties and some early implementations with no
syntactic equality (as presented in Section
\ref{sct:orig_type_operations}) we found a two minor omission in the
type system of Erlang, in an operator that returned all the simple types
contained in a composite type:
\begin{enumerate}
\item Arbitrary lists didn't return as simple types the empty list and
  the nonempty list with the same contents
\item Unspecified numbers didn't break down to unspecified integers
  and unspecified floats.
\end{enumerate}

Fixing the first omission led to the detection of many loose contracts
in OTP (where nonempty list were sure to be returned whereas the
contract included the empty list as a possible return).

\subsection{Conclusion}

Using PropEr was a very creative and fun experience as ideas could be
tested quickly against the properties required.

\section{Performance issues}

Sparsely in the previous secrions we mentioned the need of limits in
the geneneration of intersections. These limits were imposed when the
analysis was under risk to become needlessly time consuming. The
recent parallelization of Dialyzer \cite{Ypatia} should push these
limits further.
\begin{description}
\item[Number of clauses:] In the calculation of the disjunctive normal
  form, in cases where too many clauses or deep nesting of branches is
  present it is possible for the normal form to have too many
  branches. A limit was put to the number of them to maintain both
  efficiency and usability, as a very long success typing would alse
  be impractical to present to the user.
\item[Size of SCCs:] The analysis of SCCs requires a separate fixpoint
  and the substitution of the \emph{dynamic constraints} mentioned in
  Section \ref{sct:intersect_constr_generation} each time using the
  latest type. This becomes impractical when the SCC is particularly
  big so another limit was placed in the size of it.
\item[Iterations in SCCs and self-recursive:] Before intersection
  types \dr's overapproximations guaranteed that fixpoint would be
  reached in a reasonable amount of iterations. This is no longer the
  case. Consider the example in Listing \ref{lst:self_rec_id_fun}. The
  iterative process will infer that the function returns 0 for input
  0, 1 for input 1 and so forth, without any reason to stop or any
  mean to find a fixpoint (previously after a few iterations both
  success types would collapse into \emph{integer}).
  \begin{console}{lst:self_rec_id_fun}{A self recursive numeric identity function}
    id(0) -> 0; id(N) -> 1 + id(0)
  \end{console}
\end{description}

In all these cases we simply skip the tranformation to the normal form
after a fixed number of iterations. This causes the success typing to
collapse as the combination of the arguments' and the return types
happens in the end (as described in Sections \ref{sct:orig_analysis}
and \ref{sct:intersect_constr_process}).

\section{Intersection analysis results}

The actual usage of the success typings for discrepancy detection
happens in a final dataflow pass on the code under inspection. We
won't go into detail here on the various warnings that may be emitted
as these are covered in detail in the relevant publications
\cite{Elli, SuccessTypings@PPDP-06,
  springerlink:10.1007/978-3-540-30477-7_7}.

The only change we implemented there is the substitution of the
generic lookup for the return type of function calls with a lookup
that takes into consideration the types of the arguments. In this way
we can easily detect discrepancies like the one in the initial example
(\ref{sct:intersection_preliminaries}).

\subsection{Generic discrepancies}

Using the extended \dr\ the results presented in Table
\ref{tab:intersection_results} were found. They are divided in the
following categories:

\begin{description}
\item[Failing calls:] These are calls that are certain to fail. This
  is usually the result of a particular combination of arguments. This
  category includes calls that are supposed to fail but no spec is
  provided so that dialyzer knows not to worry (See Listing
  \ref{lst:failing_call}.
\item[Unneeded cases:] These are \emph{case} statements that have an
  unneeded error-catching or catch-all clause (See Listing
  \ref{lst:unneeded_case}.
\item[Exit calls:] Calls that result in an \emph{erlang:exit}. These
  come from error-handling functions that do not always fail. When
  such calls are present, user should specify that the function may
  not return (See Listing \ref{lst:exit_call}). 
\item[Nonmatching clauses:] These are nonmatching clauses that are not
  catch-alls.
\item[Deriving warnings:] In some cases a root failure may cause
  several more warnings to be emitted. These are listed with this
  category. An example is functions that won't be called due to an
  error earlier in the flow of control. Fixing the root cause will
  eliminate these warnings as well.
\end{description}

\begin{fulltable}{|c|c|c|c|}{New Discrepancies in OTP Applications}{tab:intersection_results}
\hline
Application & Description & Category & Discrepancies\\
\hline
\hline
asn1 & Abstract Syntax Notation 1 tools & Exit call & 1 \\
\hline
auth & Network Authentication Server & Deriving warnings & 9 \\
\hline
edoc & Documentation generator & Failing call & 1 \\
\hline
\multirow{2}{*}{erts} & \multirow{2}{*}{Erlang Run-Time System} &
Failing calls & 2 \\
\cline{3-4}
& & Deriving warnings & 17 \\
\hline
file & File Interface Module & Deriving warnings & 25 \\
\hline
hipe & High Performance compiler & Unneeded case & 1 \\
\hline
\multirow{3}{*}{inets}& \multirow{3}{*}{Internet clients and servers} &
Failing calls & 2 \\
\cline{3-4}
& & Exit call & 1 \\
\cline{3-4}
& & Nonmatching clause & 1 \\
\hline
\multirow{2}{*}{mnesia} & \multirow{2}{*}{distributed DBMS}
 & Exit calls & 7 \\
\cline{3-4}
& & Unneeded case & 1 \\
\hline
ssh & SSH application & Unneeded case & 1 \\
\hline
\multirow{3}{*}{ssl} & \multirow{3}{*}{Interface for Secure Socket Layer} &
Failing calls & 2 \\
\cline{3-4}
 & & Unneeded case & 1 \\
\cline{3-4}
& & Deriving warnings & 1 \\
\hline
\multirow{2}{*}{tv} & graphical examination of & \multirow{2}{*}{Unneeded case} & \multirow{2}{*}{1} \\
& ETS and Mnesia tables & & \\
\hline
\hline
\multicolumn{3}{|c|}{\textbf{Total original errors}} & 22 \\
\hline
\end{fulltable}


\begin{console}{lst:failing_call}{A call that will surely fail}
httpd_request_handler.erl:439: The call
httpd_response:send_status(ModData::#mod{data::[], method::[any()],
  request_line::nonempty_maybe_improper_list(),
  parsed_header::[any()], connection::boolean()},501, [1..255,...])
will never return since it differs in the 2nd and/or 3rd argument from
the success typing arguments: (#mod{socket_type::'ip_comm' |
  {'essl',_} | {'ossl',_} | {'ssl',_}},100 | 304 | 400 | 408 | 413 |
416 | 500 | 503,any()) or (#mod{socket_type::'ip_comm' | {'essl',_} |
  {'ossl',_} | {'ssl',_}},301 | 403 | 404 | 414,[any()]) or
(#mod{socket_type::'ip_comm' | {'essl',_} | {'ossl',_} |
  {'ssl',_}},400 | 401 | 412,'none') or (#mod{socket_type::'ip_comm' |
  {'essl',_} | {'ossl',_} | {'ssl',_}},501,{atom() |
  [any()],[any()],[any()]})

%% A rather esoteric warning about the 501 Not Implemented HTTP status message
\end{console}

\begin{console}{lst:unneeded_case}{A redundant catch-all clause}
con_desc(E) ->
    case cerl:type(E) of
	cons -> {?cons_id, 2};
	tuple -> {?tuple_id, cerl:tuple_arity(E)};
	binary -> {?binary_id, cerl:binary_segments(E)};
	literal ->
	    case cerl:concrete(E) of
		[_|_] -> {?cons_id, 2};
		T when is_tuple(T) -> {?tuple_id, tuple_size(T)};
		V -> {?literal_id(V), 0}
	    end;
	_ ->
	    throw({bad_constructor, E})
    end.

%% Produces the warning

cerl_pmatch.erl:338: The variable _ can never match since previous clauses
completely covered the type 'binary' | 'cons' | 'literal' | 'tuple'
\end{console}

\begin{console}{lst:exit_call}{A path that ends in a call to \emph{exit}}
...
check_if_valid_tag(<<>>, _, OptOrMand) ->
    check_if_valid_tag2(false,[],[],OptOrMand);
...

check_if_valid_tag2(_Class_TagNo, [], Tag, MandOrOpt) ->
    check_if_valid_tag2_error(Tag,MandOrOpt);

...

check_if_valid_tag2_error(Tag,mandatory) ->
    exit({error,{asn1,{invalid_tag,Tag}}});
check_if_valid_tag2_error(Tag,_) ->
    exit({error,{asn1,{no_optional_tag,Tag}}}).

%% Produces the warning

asn1rt_ber_bin.erl:2200: The call
asn1rt_ber_bin:check_if_valid_tag2('false',[],[],OptOrMand::any()) will never
return since it differs in the 2nd argument from the success typing arguments:
('false' | {'APPLICATION',_} | {'CONTEXT',_} | {'PRIVATE',_} |
{'UNIVERSAL',_},nonempty_maybe_improper_list(),[] | {_,_,_},any())
\end{console}

\subsection{Behaviour related results}

The introduction of intersection types removed all the false positives
from the previously collected results on behaviour usage. The reason
for this is precisely the one described in Section
\ref{sct:behaviour_discrepancies}: intersectioned types were assigned
to the functions that handled all the requests and the analysis was
able to discern whether a particular call could end in each result
instead of assuming they were all possible. None of the other warnings
were affected.

\subsection{Bonus results}

In Section REF we mentioned how the correction of a small omission
produced warnings about overspecified functions. Another such small
error regarded the relation between the \none\ and \emph{unit}
types. The infimum of the two was considered to be \emph{unit}. The
correction of this error unearthed a heap of \emph{``Function X has no
  local return''}.
